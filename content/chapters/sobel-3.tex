\chapter{Comparing Our Contrasting-Domain Model with Ippolito's (2020) Specificity-Based Model}\labch{ippolito}
How does the model we constructed in \refsec{contrast-vss} compare with other state-of-the-art accounts of reverse Sobel sequence felicity and infelicity? One of the most recent, more successful, and independently motivated attempts to accurately account for the felicity distribution of reverse Sobel sequences was proposed by \textcite{Ippolito2020}: She follows \citepos{Singh2008} original idea in proposing an overarching, pragmatic explanation for the infelicity of reverse Sobel sequences and the unidirectionality or universal infelicity of some disjunctive sequences.

Here, \textcite{Ippolito2020} distinguishes between two kinds of disjunctions: The first type consists of disjunctions where at least one of the non-scalar disjuncts entails the other and which are considered to be universally infelicitous, as shown in \refex{hurford}.
\pex[nopreamble=true]\phantomsection\label{ex:hurford}
\a\phantomsection\ljudge{\#}John is from Rome or Italy.
\a\phantomsection\ljudge{\#}John is from Italy or Rome.
\xe
These are referred to as \textit{Hurford disjunctions}---as the observation is due \textcite{Hurford1974}---and gave rise to the proposed \textit{Hurford's constraint}, as it is defined in \refdef{hurford}.
\ex\phantomsection
\extitle{Hurford’s Constraint (HC)}
A disjunction of the form $X_1\lor X_2$ is odd if $X_1$ entails $X_2$ or vice versa.\\\emptyfill\parencite[p. 202]{Katzir2014}\labdef{hurford}
\xe

The second type consists of disjunctions where only one of the disjunct entails the other but where the disjuncts are scalar terms. Here, \textcite{Gazdar1979} observed that such constructions are typically felicitous and thereby invalidate Hurford's constraint, as evidenced by \refex{scalardisjunction1-good} and \refex{scalardisjunction2-good}. As the original observation here is due \textcite{Gazdar1979}, these kinds of disjunctive sequences are also known as \enquote{Gazdar's disjunctions}.
\pex[nopreamble=true]\phantomsection\label{ex:scalardisjunction-good}
\a\phantomsection John ate some of the cookies or all of them.\labex{scalardisjunction1-good}
\a\phantomsection Mafalda will invite Felipe or [Felipe and Susanita].\labex{scalardisjunction2-good}
\xe
However, \textcite{Singh2008} notes that, similarly to the case of reverse Sobel sequences, this felicity is unidirectional: If the disjunction is to be felicitous, the entailed disjunct must be the first disjunct in the sequence: If the entailed disjunct is the second disjunct in the sequence, the disjunction is considered infelicitous. Examples of this are shown in \refex{scalardisjunction1-bad} and \refex{scalardisjunction2-bad}.
\pex[nopreamble=true]\phantomsection\label{ex:scalardisjunction-bad}
\a\phantomsection\ljudge{\#}John ate all of the cookies or some of them.\labex{scalardisjunction1-bad}
\a\phantomsection\ljudge{\#}Mafalda will invite [Felipe and Susanita] or Felipe.\labex{scalardisjunction2-bad}
\xe
Here, the parallelism between \refex{scalardisjunction-good}+\refex{scalardisjunction-bad} and \refex{SS-nuclear}+\refex{rSS-nuclear}, repeated below as \refex{yetanotherusrss}, gave rise to \citepos{Singh2008} proposal to treat these disjunctions and reverse Sobel sequences as being rendered infelicitous by a single, underlying pragmatic mechanism.
\pex[nopreamble=true]\phantomsection\label{ex:yetanotherusrss}
\a\phantomsection\labex{SS-nuclear-repeat1000}If the USA threw its weapons into the sea tomorrow, there would be war;\linebreak but if the USA and the other nuclear powers all threw their weapons into the sea tomorrow, there would be peace.\hfill\parencite[p. 10]{Lewis1973}
\a\phantomsection\labex{rSS-nuclear-repeat1000}If the USA and the other nuclear powers all threw their weapons into the sea tomorrow, there would be peace; \#but if the USA threw its weapons into the sea tomorrow, there would be war.\hfill\parencite{Heim1994}
\xe
\textcite{Ippolito2020} put forth one such a unified pragmatic proposal. We summarise the specifics of her proposal's framework in \refsec{ippolito-frame}, before applying it to the disjunctive sequences in \refsec{ippolito-disjunction} and to reverse Sobel sequences in \refsec{ippolito-rss}. Finally, we explore how her account compares to our proposal from \refsec{contrast-vss} in terms of empirical coverage and accuracy in \refsec{ippolito-comparison} before attempting to improve upon \citepos{Ippolito2020} account in \refsec{awesome}.

\section{Basic Framework}\labsec{ippolito-frame}
\citepos{Ippolito2020} account rests upon five vital pillars: (i) The way she forms structured sets of alternatives, (ii) how she determines which propositions and alternatives are made salient by the discourse, (iii) that sequences of sentences belonging to the same structured set of alternatives are subject to a specificity constraint, (iv) that an overt violation of the aforementioned specificity constraint can be avoided by covertly strengthening the weaker alternative, and (v) that covert strengthening is subject to an economy constraint s.t. covert strengthening is only a valid option if the result of said strengthening is not already part of the previous utterance's salient alternatives. We will cover each of the aforementioned pillars in turn.

\subsection{Alternatives and Structured Sets of Alternatives}\labsec{ippolito-alternatives}
First, let us explore how \textcite{Ippolito2020} views alternatives. Here, she adopts well-accepted ideas from the literature on the semantics of focus \parencite{Rooth1992,Rooth1996}, questions under discussion \parencite{Buring2003,Roberts1996}, and questions \parencite{Groenendijk1999} and takes alternatives to (i) be possible answers to a question under discussion and (ii) requiring a focus feature on some constituent at some syntactic level of representation to be constructed. An example: Let us take \enquote{John read some\textsubscript{F} books.} to be the sentence $S$. Here, focus on the quantifier \textit{some} determines that all alternatives to $S$ must be a possible answer to the underlying question under discussion of \enquote{How many books did John read?} that is constructed by replacing \textit{some} in $S$ with an expression $\beta$ in the focus value of \textit{some}. The focus value of \textit{some}, in turn, is determined by her to be a set that contains expressions whose denotations have the same semantic type as the denotation of \textit{some}; namely \textit{some}, \textit{all}, \textit{no}, and \textit{some but not all}.\footnote{Note that \textcite{Ippolito2020} makes no commitment on how the set of alternatives is constrained to just \textit{some} and \textit{all} for the calculation of quantity implicatures. She considers---and we would agree---the so-called \textit{symmetry problem} of focus semantics to be orthogonal to the question of what generally counts as an alternative, and refers to independent attempts at a solution to this problem \textcite[e.g.,][]{Katzir2007,Fox2011}.\label{footnote:symmetry}}

Adding to a hereto fairly standard Roothian account of focus semantics, \textcite{Ippolito2020} assumes the existence of structured sets of alternatives that codify the logical relation amongst the generated alternatives. \textcite{Ippolito2020} defines her notion of structured sets of alternatives as follows: \enquote{For a given sentence $S$ of the form $[_S\ldots\alpha_F\ldots]$, where $\alpha_F$ is a focused constituent in $S$, $T_{\mathbbm{A}_\alpha}$ is a \textit{structured set of alternatives} for $\alpha$ iff (i) $\mathbbm{A}_\alpha=\{\beta:\intension[]{$\beta$}\in\intension[f]{$\alpha$}\}$ (where $\intension[f]{$\alpha$}$ is the focus value of an expression $\alpha$, and for any $x,y\in\intension[f]{$\alpha$}, x,y\in D_\tau$), and (ii) $T_{\mathbbm{A}_\alpha}$ satisfies \textit{Strength}, \textit{Disjointness}, and \textit{Exhaustivity} [\dots].}~\parencite[p. 640]{Ippolito2020} These three conditions, in turn, are defined as follows by \textcite{Ippolito2020}:
\pex\phantomsection\label{def:alternativeconstruction}
\pextitle{Well-Formedness Conditions for Structured Sets of Alternatives}\\
$T_\mathbbm{A}$ is well-formed iff all of the following conditions are met:
\a\phantomsection \pextitle{Strength:} for any two alternatives $\alpha,\beta\in\mathbbm{A}$, $\beta$ is the daughter of $\alpha$ in $T_\mathbbm{A}$ just in case $\intension[]{$\beta$}\subset\intension[]{$\alpha$}$.
\a\phantomsection \pextitle{Disjointness:} for any two alternatives $\beta_1,\beta_2\in\mathbbm{A}$, if $\beta_1$ and $\beta_2$ are sisters in $T_\mathbbm{A}$, then $\intension[]{$\beta_1$}\cap\intension[]{$\beta_2$}=\emptyset$.
\a\phantomsection \pextitle{Exhaustivity:} for any alternative $\alpha$ with daughters $\beta_1,\ldots,\beta_n$ in $T_\mathbbm{A}$, $\intension[]{$\beta_1$}\cup\ldots\cup\intension[]{$\beta_n$}=\intension[]{$\alpha$}$.\hfill\parencite[p. 640]{Ippolito2020}
\xe
In more prosaic form, the condition of strength stipulates that an alternative is considered a daughter node of another alternative iff the former is a proper subset of the latter (i.e.,~they may not be identical and the daughter alternative's proposition must only consist of a fraction of the mother node's proposition's worlds). The condition of disjointness stipulates that any two direct daughters to the same mother node must consist of propositions that do not share any of the same worlds. Lastly, exhaustivity stipulates that the denotation of an alternative mother node is the union of all of its daughter nodes' denotations. Together, all of these assumptions ensure that any structured set of alternatives covers the entirety of logical space. It also ensures that any two direct sibling nodes possess the same degree of granularity and specificity and that the weakest propositions are placed at the top of the hierarchy (as a mother node would always be a logically weaker proposition than its daughter nods).

To exemplify how this derives a structured set of alternatives, reconsider our example \enquote{John read some\textsubscript{F} books}: The set of alternatives generated by \textit{some} is as follows: $\{\intension[]{some},\intension[]{all},\intension[]{no},\intension[]{some but not all}\}$. Since $\intension[]{some}$ and $\intension[]{no}$ are direct antonyms, $\intension[]{some}\cap\intension[]{no}=\emptyset$, fulfilling the condition of disjointness, thereby ensuring that they are in a lateral relation to one another. As $\intension[]{some}\cup\intension[]{no}=D_s$ (i.e.,~the union of the two denotations is equal to the set of all possible worlds), also fulfils the condition of exhaustivity, we can conclude that $\intension[]{some}$ and $\intension[]{no}$ are direct sister nodes and at the top of the structured set of alternatives' hierarchy. As $\intension[]{all}\subset\intension[]{some}$ and $\intension[]{some but not all}\subseteq\intension[]{some}$, fulfilling the condition of strength, we can establish $\intension[]{all}$ and $\intension[]{some but not all}$ as daughter nodes to $\intension[]{some}$. Since $\intension[]{all}\cap\intension[]{some but not all}=\emptyset$ and $\intension[]{all}\cup\intension[]{some but not all}=\intension[]{some}$, we can conclude that $\intension[]{all}$ and $\intension[]{some but not all}$ are valid sister nodes directly subordinate to the mother node of $\intension[]{some}$. The resulting structured set of alternatives may be visually represented as \reffig{ippolito-someall}.
\begin{figure}[!htb]
    \centering\hspace{-7cm}
    \Tree [.{} [.some {all} {some and not all} ] no ]
    \caption{The structured set of alternatives generated by \textit{some\textsubscript{F}} according to \textcite{Ippolito2020}.}
    \labfig{ippolito-someall}
\end{figure}

\noindent Overall, the structured set of alternatives in \reffig{ippolito-someall} may be considered well-formed as (i) each sister nodes fulfil the conditions of disjointness and exhaustivity and (ii) each daughter node fulfils the condition of strength in relation to its mother node.

\subsection{Salience of Alternatives}
With this, we may turn our attention to the second pillar of \citepos{Ippolito2020} proposal: How to determine which alternatives $S'$ are made salient by the utterance of any given sentence $S$. To this end, \textcite[p. 642]{Ippolito2020} defined the \textit{dynamic salience principle}, which governs which alternatives are included in the set of discourse-salient alternatives for the focus-carrying constituent $\alpha$ (symbolised as $\Delta_\alpha$). To determine which alternatives are made salient, \textcite{Ippolito2020} makes use the Stalnakerian view on the context set and how it is updated: Rather than adding worlds to the context set as the discourse continues, worlds that are incompatible with the uttered propositions are removed. E.g., our example \textit{John read some books} would eliminate from the context set all worlds where John read no books at all. As such, the complement of $\intension[]{John read some books}$, namely $\intension[]{John read no books}$ is made salient in the process of removing the latter's worlds from the context set. Incidentally, due to the condition of disjointness in structured sets of alternatives, this corresponds to the union of all of its sister nodes (or to just its sister node in case it has only one of them). Furthermore, any mother node, if present, is made salient as the context set must also be in compliance with it, since the mother node is entailed by our original utterance. As such, \textcite{Ippolito2020} first defined the dynamic salience principle as shown in \refdef{salience-simple}:
\ex\phantomsection\labdef{salience-simple}\pextitle{Dynamic Salience Principle} (to be revised)\\
Uttering a sentence $S=[_S\ldots\alpha_F\ldots]$ makes salient a subset $\Delta_\alpha$ of the set of alternatives in $T_{\mathbbm{A}_\alpha}$ such that $\Delta_\alpha=\{\beta\in T_{\mathbbm{A}_\alpha}: \beta$ is $\alpha$’s sister or $\alpha$’s mother\}.\\\emptyfill\parencite[p. 641]{Ippolito2020}
\xe

\textcite{Ippolito2020} attempted to further formalise and abstract this definition. She argues that the following definition in \refdef{salience}---where $\leqslant_\alpha$ refers to the function of \textit{logically equally close or closer to $\alpha$ than}, as it is also defined in \refdef{salience}\footnote{It should be noted that, in \citepos{Ippolito2020} paper, the function of \textit{logically equally close or closer to $\alpha$ than} is represented by $\geqslant_\alpha$. We reversed the direction of this function so that its functionality is in line with the world-closeness and world-similarity functions used throughout this dissertation.}---achieves the desired result of rendering any sister node and mother node salient. 
\ex\phantomsection
\extitle{Dynamic Salience Principle, revised}
Uttering a sentence $S=[_S\ldots\alpha_F\ldots]$ makes salient a subset $\Delta_\alpha$ of the set of alternatives in $T_{\mathbbm{A}_\alpha}$ such that $\forall\alpha'\in\Delta_\alpha[\forall\alpha''\in T_{\mathbbm{A}_\alpha}: \alpha'\leqslant_\alpha \alpha'']$, where $\forall\alpha',\alpha'':\alpha'\leqslant_\alpha \alpha''$ just in case $\{p\in\wp(W):[_S\ldots\alpha_F\ldots]\subseteq p~\&~[_{S''}\ldots\alpha''\ldots]\subseteq p\}\subseteq\{p\in\wp(W):[_S\ldots\alpha_F\ldots]\subseteq p~\&~[_{S'}\ldots\alpha'\ldots]\subseteq p\}$.\\\emptyfill\parencite[p. 642]{Ippolito2020}\labdef{salience}
\xe

Explained in a more prosaic fashion, \textcite{Ippolito2020} proposes that the following alternatives are made salient via utterance: If some utterance $S=[_S\ldots\alpha_F\ldots]$ is uttered, the generated alternatives $\alpha'$ logically closest to the original $\alpha$ are made salient, where the logical closeness of the alternatives is determined by the number of entailments that are shared between $S=[_S\ldots\alpha_F\ldots]$ and $S'=[_{S'}\ldots\alpha'\ldots]$ (i.e.,~the original utterance and the alternative utterance). As such, if the number of entailments $S'=[_{S'}\ldots\alpha'\ldots]$ shares with $S=[_{S}\ldots\alpha_F\ldots]$ is greater than the number of entailments $S''=[_{S''}\ldots\alpha''\ldots]$ shares with $S=[_{S}\ldots\alpha_F\ldots]$, then \textcite{Ippolito2020} considers $\alpha'$ to be logically closer to $\alpha$ than $\alpha''$. According to \textcite{Ippolito2020}, this definition ensures that both the sister nodes and the mother node of $\alpha$ are made salient, if such nodes exist, as any $\alpha$ has at least two entailments: That it entails itself and that, due to the condition strength, it entails its mother node. Consider the tree in \reffig{ippolito-baretree}:
\begin{figure}[!htb]
    \centering
    \hspace{-7cm}\Tree [.{$\boldsymbol{\gamma}$} {\underline{$\alpha$}} {$\boldsymbol\beta$} ]
    \caption{A nondescript, minimal structured set of alternatives as according to \textcite{Ippolito2020}. The original utterance is underlined, and all salient alternatives are in boldface.}
    \labfig{ippolito-baretree}
\end{figure}

\noindent Due to the condition of disjointness and strength, it is a certainty that any sister $\beta$ does not entail $\alpha$ and that $\beta$ also entails their shared mother node $\gamma$ (and any shared nodes above $\gamma$). As such $\beta$ should always be one entailment removed from $\alpha$. For the mother node of $\alpha$, the situation is identical, though for different reasons. Since $\alpha$ entails any mother node $\gamma$, and any node entails itself, the mother node shares the entailment of $\gamma$ with $\alpha$ (and any other nodes above $\gamma$). As such, the mother would also always be one entailment removed from $\alpha$. This would render $\beta$ and $\gamma$ the closest alternatives to $\alpha$ according to \textcite{Ippolito2020}; i.e., $\Delta_{\alpha}=\{\beta,\gamma\}$.

However, we would disagree that \citepos{Ippolito2020} definition in \refdef{salience} is functionally equivalent to \refdef{salience-simple}. There are two reasons for this: First, \textcite{Ippolito2020} did not exclude the possibility of the original utterance itself being considered as an alternative to be evaluated even though it is an alternative contained within $T_{\mathbbm{A}_\alpha}$. Since no alternative is going to have as much entailments in common with $\alpha$ as $\alpha$ itself, we would have to assume that $\Delta_\alpha=\{\alpha\}$ (i.e., neither the sister nor the mother node is salient). We would assume, of course, that \textcite{Ippolito2020} implicitly assumed for this limitation to be imposed upon the definition, and, as such, this may seem as an uncharitable and pedantic objection. However, we only noted this issue because it leads us to a less trivial objection to \refdef{salience}: If $\Delta_\alpha$ consists only of the logically closest alternatives to $\alpha$ in $T_{\mathbbm{A}_\alpha}$, then it must also follow that any and all daughter nodes of $\alpha$ are logically closer to $\alpha$ in terms of shared entailments than any sister node $\beta$ or parent node $\gamma$ (since they would entail both $\alpha$ and $\gamma$). As such, if $\alpha$ has any daughter nodes at all, this would prevent the desired sister and parent nodes to rise to salience. As such, this must also be accounted for. To this effect, we would propose the added restriction that any alternatives that logically entail $\alpha$ are also excluded from the comparison function. To this end, we would propose the following more restrictive definition of the dynamic salience principle in \refdef{salience-my}:
\ex\phantomsection
\extitle{Dynamic Salience Principle, revised and restricted}
Uttering a sentence $S=[_S\ldots\alpha_F\ldots]$ makes salient a subset $\Delta_\alpha$ of the set of alternatives in $T_{\mathbbm{A}_\alpha}$ such that $\forall\alpha'\in\Delta_\alpha[\forall\alpha''\in T_{\mathbbm{A}_\alpha}: \alpha'\leqslant_\alpha \alpha''$ and $[_{S}\ldots\alpha_F\ldots]\not\supseteq[_{S'}\ldots\alpha'\ldots]]$, where $\forall\alpha',\alpha'':\alpha'\leqslant_\alpha \alpha''$ just in case $\{p\in\wp(W):[_S\ldots\alpha_F\ldots]\subseteq p~\&~[_{S''}\ldots\alpha''\ldots]\subseteq p\}\subseteq\{p\in\wp(W):[_S\ldots\alpha_F\ldots]\subseteq p~\&~[_{S'}\ldots\alpha'\ldots]\subseteq p\}$.\labdef{salience-my}
\xe
With this, the sister and mother nodes are the only alternatives that are made salient. 


\subsection{Specificity Constraint}\labsec{specificity}
As such, we may turn our attention to the third pillar of \citepos{Ippolito2020} framework: the specificity constraint she lays upon sequences of sentences that have the form of $<[_{S_1}\ldots\alpha_F\ldots],[_{S_2}\ldots\beta_F\ldots]>$, where $\beta\in T_{\mathbbm{A}_\alpha}$, and where $S_1$ and $S_2$ both answer the same question under discussion. According to \textcite{Ippolito2020}, a sequence of this type is intuitively expected to contain sentences that are maximally informative with respect to one another (i.e., the are equal in specificity). Naturally, $\alpha$ and $\beta$ would be maximally informative to one another if they are disjoint. Due to the condition of disjointness, this characterises propositions that are sister nodes in a structured set of alternatives. As such, \textcite{Ippolito2020} proposes the specificity condition in \refdef{specificity} to codify, in her framework of structured sets of alternatives, that sentences in such sequences must be maximally informative by being disjoint to one another.
\pex\phantomsection
\extitle{Specificity Condition}\labdef{specificity}
A sequence $\Sigma<[_{S_i}\ldots\alpha_F\ldots],[_{S_j}\ldots\beta_F\ldots]>$, s.t. both
$S_i$ and $S_j$ are answers to the same QUD and $\beta\in T_{\mathbbm{A}_\alpha}$, is felicitous if either \dots
\a\phantomsection \dots $\alpha$ or $\beta$ is the only node on its branch in $T_{\mathbbm{A}_\alpha}$, or \dots \labdef{specificity-klecharelevant}
\a\phantomsection \dots $\alpha$ and $\beta$ are dominated by the same number of nodes in $T_{\mathbbm{A}_\alpha}$.\\\emptyfill\parencite[p.~643]{Ippolito2020}\labdef{specificity-sister}
\xe
With this, the denotations of the sequence's sentences are ensured to be maximally informative---i.e., disjoint---to one another, as \refdef{specificity-sister} ensures that the two denotations are either (i) sisters and thereby disjoint or (ii) cousins of the same level of embeddedness elsewhere in the structure and thereby also disjoint due to the fact that some of their mother nodes (or above) must be sisters to one another. The condition in \refdef{specificity-klecharelevant} achieves the same result but allows for greater leniency as to the relationship between two sentences if one of them is the sole member of its branch in the structured set of alternatives. We further explore the reason for its existence in \refsec{ippolito-disjunction} and \refsec{ippolito-rss}.

\subsection{Covert Strengthening and Economic Constraint}
Finally we may turn to the last two pillars of \citepos{Ippolito2020} framework: the covert strengthening of propositions and the economic constraint placed upon cover strengthening operations. \textcite{Ippolito2020} stipulates that, if some sequence $\Sigma=<S_i,S_j>$ appears to overtly violate the specificity condition introduced in \refsec{specificity}, we attempt to covertly strengthen the weaker member of the sequence via some operator in an attempt to satisfy the specificity condition. However, this covert strengthening is only a valid option if the result of said strengthening is not equivalent to some item in the set of salient discourse alternatives ($\Delta$) (i.e., that no previous utterance has already made the overt equivalent of the covert strengthening process salient). If that was the case, \textcite{Ippolito2020} argues that the covert strengthening operation would be too costly, due to the overt equivalent already being in the forefront of our minds and must therefore be preferred to the alternative that requires covert strengthening. In such a scenario, choosing the overtly weaker alternative results in infelicity. This was codified by \textcite{Ippolito2020} in \refdef{economy}:
\ex\phantomsection\labdef{economy}
\extitle{Economy}
For any sequence $\Sigma=<[_{S_{n-1}}\ldots\alpha_F\ldots],[_{S_{n}}\ldots cs(\beta_F)\ldots]>$, where $cs$ is a covert strengthening operator and $\beta\in T_{\mathbbm{A}_\alpha}$:\\
$\#\Sigma$ if $\exists\gamma\in\Delta_\alpha$ s.t. the sentence $S'_n$ obtained by replacing $cs(\beta)$ in $S_n$ with $\gamma$ is equivalent to $S_n$.\hfill\parencite[p.~643]{Ippolito2020}
\xe
It is this principle of economy that \citepos{Ippolito2020} main factor for deciding on and deriving the infelicity of sequences---disjunctive or conditional.

Two things are of note in \refdef{economy}: First, \textcite{Ippolito2020} wrote that the covert strengthening operator is directly applied to $\beta$, but uses this as shorthand for $cs$ being applied to the smallest sentential node in the utterance's LF that contains $\beta$---as $cs$-operators are sentential operators and therefore require some proposition $\phi\in D_{\langle s,t\rangle}$ as input. Second, it should also be noted that the covert strengthening operator in \refdef{economy} is intentionally kept abstract as \textcite{Ippolito2020} argues that there are two different strengthening mechanisms at play for disjunctive sequences and conditional sequences, as is discussed, respectively, in \refsec{ippolito-disjunction} and \refsec{ippolito-rss}. 

\section{Disjunctive Sequences}\labsec{ippolito-disjunction}
Having established \citepos{Ippolito2020} framework, we may turn our attention to how it derives the infelicity of infelicitous disjunctive sequences. Here, there are two types of sequences to account for: Hurford disjunctions (which are the universally infelicitous disjunctive sequences) and Gazdar's disjunctions (which are the unidirectional disjunctive sequences).

Let us consider the traditional Hurford disjunctions in \refex{hurford}, repeated in \refex{hurford-repeat1}:
\pex[nopreamble=true]\phantomsection\label{ex:hurford-repeat1}%
\a\phantomsection\ljudge{\#}John is from Rome or Italy.
\a\phantomsection\ljudge{\#}John is from Italy or Rome.
\xe
Assuming that the disjuncts carry focus, \textcite{Ippolito2020} would construct the same structured set of alternatives for either of the Hurford disjunctive sequence in \reffig{ippolito-romeitaly}.
\begin{figure}[!htb]
    \centering\hspace{-2cm}
    \Tree [ [.{Europe} [.Italy {Rome} {Milan} {\dots} ] [.France {Paris} {Marseille} {\dots} ] {\dots} ] {\ldots} ]
    \caption{The structured set of alternatives generated by either \textit{John is from Rome or Italy} or by \textit{John is from Italy or Rome} according to \textcite{Ippolito2020}}
    \labfig{ippolito-romeitaly}
\end{figure}
Here, it is obvious that either direction would overtly violate the specificity constraint in \refdef{specificity}. As such, either direction would predicted to be infelicitous unless covert strengthening may bring at least on of the sequences to conformity with the specificity constraint---whilst adhering to the economic constraint in \refdef{economy}. As such, we must ask ourselves what covert strengthening operators may take place here: \textcite{Ippolito2020} assumes that the covert strengthening operator in typical declaratives such as in disjunctive sequences corresponds to the covert exhaustivity operator {\scshape exh} proposed by \textcite{Chierchia2012}, as defined in \refdef{exh}, repeated below as \refdef{exh-repeat}.
\ex\phantomsection\labdef{exh-repeat}$\intension[g,c]{\scshape{exh}}(C)(\phi)(w)=\textiff\phi(w)=1\land\forall\psi\in{C}[\psi\not\subseteq\phi\land\psi(w)=0]$%
\xe
That is to say, the covert exhaustification of any proposition $\phi$ asserts the implicature that any alternative $\psi$ that is not entailed by $\phi$ is negated (that is, if the focused constituent of $\phi$ corresponds to a scalar term, as exhaustification requires scalarity). Here, it should be noted that the negated alternatives correspond to only a subset of what \textcite{Ippolito2020} considers to be all possible alternatives (as described in \refsec{ippolito-alternatives})---related to the symmetry problem in alternative semantics (as referenced and given thought to in Footnote~\ref{footnote:symmetry} on Page~\pageref{footnote:symmetry}). For the purpose of this chapter, we make the simplistic assumption that {\scshape exh} negates only atomic alternatives (e.g., it would negate \textit{all} but not \textit{some but not all}, as shown later).

Now, the issue with the Hurford disjunctions in \refex{hurford} is that they do not contain any scalar term and as such are not eligible for covert exhaustification. As such, the logically weaker alternative cannot be strengthened, resulting in an inevitable violation of the specificity constraint, resulting in the infelicity of any Hurford disjunction.

With this, we turn our attention to the unidirectional, scalar Gazdar's disjunctions in \refex{scalardisjunction-good} and \refex{scalardisjunction-bad}, repeated below as \refex{scalardisjunction-good-repeat1} and \refex{scalardisjunction-bad-repeat1}, respectively.
\pex[nopreamble=true]\phantomsection\label{ex:scalardisjunction-good-repeat1}%
\a\phantomsection John ate some of the cookies or all of them.\labex{scalardisjunction1-good-repeat1}
\a\phantomsection Mafalda will invite Felipe or [Felipe and Susanita].\labex{scalardisjunction2-good-repeat1}
\xe
\pex[nopreamble=true]\phantomsection\label{ex:scalardisjunction-bad-repeat1}%
\a\phantomsection\ljudge{\#}John ate all the cookies or some of them.\labex{scalardisjunction1-bad-repeat1}
\a\phantomsection\ljudge{\#}Mafalda will invite [Felipe and Susanita] or Felipe.\labex{scalardisjunction2-bad-repeat1}
\xe
Starting with \refex{scalardisjunction1-good-repeat1},  and assuming that each disjunct's scalar term carries focus (i.e.,~\refex{scalardisjunction1-good-repeat1} is rendered as \textit{John ate some\textsubscript{F} cookies or all\textsubscript{F} of them}), \textcite{Ippolito2020} would assume the structured set of alternatives in \reffig{ippolito-someall-nosalience}.
\begin{figure}[!htb]
    \centering\hspace{-6cm}
    \Tree [.{} [.some {all} {some and not all} ] no ]
    \caption{The structured set of alternatives generated by either \textit{some\textsubscript{F}} or \textit{all\textsubscript{F}} according to \textcite{Ippolito2020}.}
    \labfig{ippolito-someall-nosalience}
\end{figure}
Here, \textit{some} and \textit{all} are neither the only node on their branch in their structured set of alternatives nor are they dominated by the same number of nodes. As such, \refex{scalardisjunction1-good-repeat1} would overtly fail to adhere to the specificity condition. To rectify this, we may attempt to covertly strengthen the logically weaker disjunction. Since the weaker disjunct in \refex{scalardisjunction1-good-repeat1} is context-initial, it is guaranteed that any strengthening would not violate any economic constraints, as the salient set of alternatives is empty due to a lack of relevant previous utterances. Since $\Delta_\emptyset=\emptyset$, no strengthened alternative can be identical with any alternative in $\Delta_\emptyset$, fulfilling the principle of economy in \refdef{economy}. If \textit{some} is strengthened via covert exhaustification, the alternative of \textit{all} is negated, resulting in the strengthened reading of \textit{some and not all}. As such, the sequence would read as $\Sigma_\text{\refex{scalardisjunction1-good-repeat1}}=<\text{{John ate some and not all cookies}},\text{{John ate all cookies}}>$, which no longer violates the specificity constraint in \refdef{specificity}, as both alternatives are dominated by the same number of nodes, as is shown in \reffig{ippolito-someall-nosalience}.

What if we were to reverse the sequence, as in \refex{scalardisjunction1-bad-repeat1}? In that case, the logically weaker term \textit{some} is no longer faced with an empty set of salient alternatives and, as such, may be subject to a violation of economy. To check whether or not this is the case, we must identify which alternatives are made salient by the utterance of \textit{John ate all of the cookies}. To this end, we must identify the logically closest alternatives (that do not entail the original) in the terms of shared entailments, as detailed in \refdef{salience-my}. In our case, \textit{all} entails two alternatives: \textit{some} and \textit{all}. The alternatives \textit{some and not all} and \textit{some} each share the former entailment and are, therefore, one entailment removed from the \textit{all}, making them the closest alternatives, raising them to salience, as shown in \reffig{ippolito-someall-salience}.
\begin{figure}[!htb]
    \centering\hspace{-6cm}
    \Tree [.{} [.\textbf{some} {\underline{all}} {\textbf{some and not all}} ] no ]
    \caption{The structured set of alternatives generated by \textit{all\textsubscript{F}} according to \textcite{Ippolito2020}. The original utterance is underlined, and all salient alternatives are in boldface.}
    \labfig{ippolito-someall-salience}
\end{figure}

\noindent As such, since $\Delta_\text{all}=\{\text{some},\text{some and not all}\}$, the covert strengthening of \textit{some} to \textit{some and not all} via exhaustification would violate the principle of economy, making it too costly in face of its overt alternative, rendering the disjunctive sequence in \refex{scalardisjunction1-bad-repeat1} infelicitous. This difference in the set of salient alternatives and its corresponding impact on the economy of covert strengthening is what explains the unidirectionality of Gazdar's disjunctive sequences in \citepos{Ippolito2020} framework. The explanation for \refex{scalardisjunction2-good-repeat1} and \refex{scalardisjunction2-bad-repeat1} are perfectly analogous, as \textcite{Ippolito2020} considers \textit{Felipe} and \textit{Felipe and Susanita} to be part of a conjunctive scale, making the sentence subject to covert exhaustification.

Moving away from Hurford disjunctions and Gazdar's disjunctions, there are two noteworthy types of disjuncts that \textcite{Ippolito2020} also accounts for. First, \citepos{Ippolito2020} framework makes sure to account for why \textit{no} may be partnered with any other scalar term in a disjunctive sequence, as shown in \refex{ippolito-showoff}:
\pex[nopreamble=true]\phantomsection\label{ex:ippolito-showoff}%
\a\phantomsection John ate no cookies or all of them.
\a\phantomsection John ate all of the cookies or none of them.
\a\phantomsection John ate no cookies or some of them.
\a\phantomsection John ate some cookies or none at all.
\xe
This is accomplished with the hereto unused condition for specificity in \refdef{specificity-klecharelevant} that allows for two items to not be dominated by the same number of nodes in the structured set of alternatives under singular condition that either of the sequence's members is the only node on its branch. This is the case for \textit{no}, as easily becomes apparent consulting either \reffig{ippolito-someall-nosalience} or \reffig{ippolito-someall-salience}.

The second other type of disjunction that \textcite{Ippolito2020} may account for are vaguely related to Gazdar's disjuncts: Disjunctive sequences that do not entail each other but are also not equal in granularity \parencite[p.~648, Footnote~12]{Ippolito2020}. See the following example discourse in \refex{ippolito-showoff2}:
\pex\phantomsection\label{ex:ippolito-showoff2}%
\speaker{Q}~\enquote{Where does John come from?}
\a\phantomsection \speaker{A\textsubscript{1}}{\#}He is from Rome or France.
\a\phantomsection \speaker{A\textsubscript{2}}{\#}He is from France or Rome.
\xe
Here, the disjuncts answer the same question under discussion but they do not entail one another and they are not dominated by an equal number of nodes. As such, they would violate the specificity constraint's condition in \refdef{specificity-sister}, correctly predicting that these sequences should be infelicitous.

\section{(Reverse) Sobel Sequences}\labsec{ippolito-rss}
Having accounted for the felicity and infelicity of Gazdar's disjunctions as well as the universal infelicity of Hurford's disjunctions, we may turn our attention to how \textcite{Ippolito2020} accounts for the felicity of Sobel sequences and for the varying degrees of (in-)felicity of reverse Sobel sequences.

In order to see how \textcite{Ippolito2020} accounts for reverse Sobel sequences, we must first cover how she handles the basic semantics of conditionals: First, it must be noted that \textcite{Ippolito2020} completely restricts her analysis to counterfactual conditionals and makes no definitive claims about non-counterfactuals. Second, for counterfactuals, she adopts a fairly standard variably-strict semantics for conditionals.
\ex\phantomsection\labdef{variablystrict-repeat666}For all contexts $c$, \enquote{If $\phi$, $\psi$} is true at $w$ in $c$ iff all the closest $\phi$-worlds to $w$ are $\psi$-worlds, where closeness is determined by similarity.%

\xe
However, contrary to the standard \textcite{Stalnaker1968} and \textcite{Lewis1973} framework, she considers the world similarity function to be relativised to the focus value of a conditional's antecedent (i.e., the similarity ranking only takes worlds into account that correspond to the alternatives of the antecedent currently being evaluated). To this end, she constructs the set of propositions $\mathbbm{P}$ that is defined as equal in meaning to the focus value of the antecedent $\phi$ containing the focused constituent $\alpha_F$, as shown in \refdef{ippolito-relativealts}, and relativises the similarity ranking function $sim_{\leqslant,w_c}$, that rates how similar worlds are to the actual world $w_c$, to aforementioned set of propositions as $sim_{\leqslant,w_c,\mathbbm{P}_\phi}$, as defined in \refdef{ippolito-similarity}.
\pex[nopreamble=true]\phantomsection\label{def:ippolito-vs}%
\a\phantomsection $\mathbbm{P}_{[_\phi\ldots\alpha_F\ldots]}=\intension[f]{$[_\phi\ldots\alpha_F\ldots]$}=\{\psi:\exists{x}[x\in\mathbbm{A}_\alpha\land\psi=\lambda{w}.[\ldots{x}\ldots]]\}$\labdef{ippolito-relativealts}
\a\phantomsection $sim_{\leqslant,w_c,\mathbbm{P}_\phi}(\phi)=\\$\emptyfill$\{w:\phi(w)=1\land\forall{w'}[\exists\psi\in\mathbbm{P}_\phi[\psi(w')=1\land\phi(w')=1]\rightarrow w\leqslant_{w_c}w']\}$\labdef{ippolito-similarity}
\xe
Essentially, \citepos{Ippolito2020} variably-strict account is identical to the traditional accounts from \textcite{Stalnaker1968} and \textcite{Lewis1973} with the exception of what worlds are being compared in similarity relative to the evaluation world. Where the traditional account select the closest antecedent worlds as compared to all other possible worlds, \textcite{Ippolito2020} selects the closest antecedent worlds as compared to the subset of all possible worlds that (i) answer the same conditional question under discussion and (ii) correspond to some of the alternative antecedent values generated by the original antecedent that is being evaluated. The reason for this restriction is mostly due to \citepos{Ippolito2020} need for parallelism between the covert strengthening via exhaustification and what she pictures as covert strengthening via the world similarity restriction of conditionals.\footnote{After all, the relativisation of the similarity comparison function to the alternatives generated by the antecedent would have no consequence on the final selection of the closest antecedent world since this relativisation does not affect the ordering of worlds itself but merely restricts the ordering to a subset of all possible worlds. This subset must, by necessity, contain the objectively closest antecedent world in general (as this is guaranteed to be an alternative to the antecedent), ensuring its selection either way. As such, it is not a process required for correct predictions concerning the evaluation of conditionals. It is nevertheless not an unrealistic proposal, especially from a psycholinguistic economy perspective. It seems far more realistic to us that an antecedent is only compared with respect to related worlds rather than being compared to all possible worlds.} The relativisation of the similarity comparison function achieves this parallelism in the following manner: By restricting the world similarity ranking to the alternatives of the antecedent which answer the same conditional question under discussion, the variably-strict semantics chooses the logically strongest reading available that still conforms to the antecedent by eliminating all unnecessary but possible deviance to the evaluation world. In this sense, this is identical to the process of covert exhaustification of focused elements in disjunctive sequences, where the weak reading was covertly strengthened to the logically stronger alternative---in comparison to the alternatives generated by the disjuncts that answer the same question under discussion.

We demonstrate this with how \textcite{Ippolito2020} accounts for Sobel sequences and reverse Sobel sequences, comparing it to a functionally identical disjunctive sequence.

\pex[nopreamble=true]\phantomsection%
\a\phantomsection If [the USA]\textsubscript{F} had thrown their weapons into the sea, there would have been war.
\a\phantomsection But if [all nuclear powers]\textsubscript{F} had thrown their weapons into the sea, there would have been peace.
\xe

\ex\phantomsection
Either [the USA]\textsubscript{F} has thrown its weapons into the sea and there is war or [all nuclear powers]\textsubscript{F} have thrown their weapons into the sea and there is peace.
\xe

Now, \citepos{Ippolito2020} framework would create the structured set of alternatives $T_{\mathbbm{A}_\text{USA}}$ in \reffig{ippolito-usa} for either the conditional sequence or the disjunctive sequence, though to different questions under discussion: The disjunctive sequence is likely an answer to the possible example question under discussion \enquote{What happened to the world after the US-nuclear disarmament conference?} whereas the conditional sequence answers the conditional question under discussion \enquote{If who had thrown their weapons into sea, would there have been peace?}.

\begin{figure}[!htb]
    \centering\hspace{-2cm}
    \Tree [ [.{USA} {all nuclear powers} {USA but no other nuclear power} ] {not USA} ]
    \caption{The structured set of alternatives generated by \textit{[the USA]\textsubscript{F}} according to \textcite{Ippolito2020}.}
    \labfig{ippolito-usa}
\end{figure}

For the disjunctive sequence, it is clear that \textit{the USA} must be covertly strengthened to \textit{the USA but no other nuclear power} such that it and its co-disjunct fulfill the specificity constraint of being dominated by an equal amount of nodes in the structured set of alternatives.

For the conditional sequence, \textcite[p. 651]{Ippolito2020} reasons as follows: In a variably-strict semantics, the similarity function restricts the evaluated worlds to those that introduce the least amount of change to the evaluation world whilst being antecedent worlds. Given that---as of yet---no nuclear powers have thrown their weapons into the sea tomorrow, the closest antecedent worlds would be worlds where only the USA have thrown their weapons into the sea tomorrow.\footnote{We find that conclusion debatable, as previously mentioned, as it is rather ambiguous under which conditions the USA would throw their weapons into the sea. It could be argued that it might require fewer changes to the laws and facts our worlds to imagine a world where all nuclear powers agreed to nuclear disarmament in comparison to the global superpower of our day unilaterally deciding to simply get rid of their nuclear arsenal. Either way, we continue under the assumption that \textcite{Ippolito2020} is correct for the sake of argument.} As such, the overt form of \textit{the USA} causes the similarity function to select for \textit{the USA but no other nuclear power} by virtue of maximal similarity to the evaluation world. As such, \textcite{Ippolito2020} considers the similarity function of counterfactual conditionals to be a form of covert strengthening that is subject to the same conditions as any other form of covert strengthening: i.e., being subject to the economy principle in \refdef{economy}. Naturally, as the strengthening occurs context-initially, the principle of economy is impossible to violate for the first conditional. In addition, as both conditionals answer the same conditional under discussion, this Sobel sequence is also subject to the specificity constraint. In the case of our regularly ordered Sobel sequence, with the covert strengthening via the similarity function, this constraint is adhered to, as \textit{USA but no other nuclear power} and \textit{all nuclear powers} are dominated by the same number of nodes in $T_{\mathbbm{A}_\text{USA}}$.

For reverse Sobel sequences, the reasoning is accordingly parallel to the reasoning used for the infelicitous disjunctive sequences in \refsec{ippolito-disjunction}: The infelicity is derived by the necessity for covert strengthening to satisfy the specificity constraint in a situation where doing so would violate the principle of economy, making the process too costly, causing an infelicitous reading. Consider the reverse disjunctive sequence in \refex{ippolito-reverseSSdisjunct} and compare it with the reverse Sobel sequence in \refex{ippolito-reverseSS}:
\ex\phantomsection
\ljudge{\#}Either [all nuclear powers]\textsubscript{F} have thrown their weapons into the sea and there is peace or [the USA]\textsubscript{F} has thrown its weapons into the sea and there is war.\labex{ippolito-reverseSSdisjunct}
\xe
\pex[nopreamble=true]\phantomsection\label{ex:ippolito-reverseSS}%
\a\phantomsection If [all nuclear powers]\textsubscript{F} had thrown their weapons into the sea, there would have been peace.
\a\phantomsection \ljudge{\#}But if [the USA]\textsubscript{F} had thrown their weapons into the sea, there would have been war.
\xe
For either sequence type the same reasoning holds true; the only difference between them is the type of covert strengthening, where the disjunctive sequence makes use of the {\scshape exh} operator and the conditional sequence makes use of the relativised similarity function $sim_{\leqslant,w_c,\mathbbm{P}_\phi}$. As both disjuncts and both conditionals answer the same respective question under discussion (as previously lined out for the regularly ordered sequences), either sequence type is subject to the specificity constraint---and the use of covert strengthening to the principle of economy. Let us review how \textcite{Ippolito2020} would derive the infelicity of \refex{ippolito-reverseSSdisjunct}. First, either sequence would derive the structured set of alternatives in \reffig{ippolito-usa-salience}.
\begin{figure}[!htb]
    \centering\hspace{-2cm}
    \Tree [ [.{\textbf{USA}} {\underline{all nuclear powers}} {\textbf{USA but no other nuclear power}} ] {not USA} ]
    \caption{The structured set of alternatives generated by \textit{[the USA]\textsubscript{F}} according to \textcite{Ippolito2020}. The original utterance is underlined, and all salient alternatives are in boldface.}
    \labfig{ippolito-usa-salience}
\end{figure}
Since the logically weaker disjunct is the second one, we must take into consideration which alternatives have been rendered salient by the first disjunct in $\Delta_\text{all-nuclear-powers}$. The alternative \textit{all nuclear powers} carries two entailments: First, it entails itself, and, second, it entails its mother node. The entailment of the mother node is shared by two alternatives: its sister node \textit{the USA but no other nuclear power} and its mother node. Therefore, these alternatives are made salient such that $\Delta_\text{all-nuclear-powers}=\{\text{USA},\text{USA but no other nuclear power}\}$. The covert logical strengthening of \textit{the USA} to \textit{the USA but no other nuclear power} would therefore violate the principle of economy, considering that its overt equivalent is a possibility made salient by the previous relevant discourse. As such, \textcite{Ippolito2020} would correctly predict the disjunctive sequence in \refex{ippolito-reverseSSdisjunct} to be infelicitous.

Concerning the reverse Sobel conditional sequence, the same reasoning would apply. The initial conditional would render the alternative \textit{the USA but no other nuclear power} salient. The world similarity function would, according to \textcite{Ippolito2020}, select the closest antecedent worlds for the second conditional, which are, in turn, worlds in which the USA has thrown its weapons into the sea but no other nuclear power did likewise. As such, the overt form of \textit{the USA} is strengthened to \textit{the USA but no other nuclear power} via the similarity function. As \textcite{Ippolito2020} considers the similarity function to be subject to the principle of economy, she would deem its use too costly in this instance, predicting the reverse Sobel sequence to be infelicitous.

\begin{figure}[!htb]
    \centering\hspace{-6cm}
    \Tree [ [.{$\boldsymbol{\phi}$} {\underline{$\phi\land\psi$}} {$\boldsymbol{\phi\land\neg\psi}$} ] {$\neg\phi$} ]
    \caption{The generic structured set of alternatives generated by a reverse Sobel sequence's $\phi\land\psi$-conditional and subsequent strengthened $\phi$-conditional.}
    \labfig{ippolito-generic}
\end{figure}
This reasoning may be abstracted to all (reverse) Sobel sequences, as shown in \reffig{ippolito-generic}. For any conditional sequence containing the conditionals \enquote{If $\phi$, $\chi$} and \enquote{If $\phi\land\psi$, $\neg\chi$}, their conditional question under discussion as identified by some focused constituent in their antecedents must be \enquote{If what, $\chi$?} \parencite[p. 650]{Ippolito2020} Being a sequence of conditionals that pertains to a single question under discussion, any (reverse) Sobel sequence is subject to the specificity constraint in \refdef{specificity}. Any reverse Sobel sequence that violates the specificity constraint in \refdef{specificity} or the principle of economy in \refdef{economy}---that is, covertly strengthens its $\phi$-conditional such that it is identical to an alternative made salient by the preceding $\phi\land\psi$-conditional---is rendered infelicitous. Regularly ordered cannot violate the principle of economy if they are uttered discourse-initially, and, as such, are typically rendered felicitous.


\subsection{Accounting for the Empirical Data and Comparison}\labsec{ippolito-comparison}
Having established what criteria \textcite{Ippolito2020} uses to determine the infelicity of reverse Sobel sequences, we may turn our attention to the varying factors that affect the felicity of reverse Sobel sequences. We established and summarised this in \refsec{introspection} in \reftab{ourdata}. Having established what criteria \textcite{Ippolito2020} uses to determine the infelicity of reverse Sobel sequences, we may turn our attention to the varying factors that affect the felicity of reverse Sobel sequences. We established and summarised this in \refsec{introspection} in \reftab{ourdata}, repeated below as \reftab{ourdata-repeat666}.
\begin{table}[!htb]
\caption{Current empirical data on felicity distribution, broken down by causality, counterfactuality, and overt denigration of relevance (or implicit epistemic exclusion) of $\psi$, with example numbers that exemplify each reverse Sobel sequence condition. Contrastive stress on the auxiliary verb is assumed for all reverse Sobel sequences.}
\resizebox{\textwidth}{!}{
    \begin{tabular}{lcccccccc}\toprule
                &   \multicolumn{4}{c}{Acausal}     &  \multicolumn{4}{c}{Causal}\\
                & \multicolumn{2}{c}{Non-Counterfactual}  &   \multicolumn{2}{c}{Counterfactual}    & \multicolumn{2}{c}{Non-Counterfactual}  &   \multicolumn{2}{c}{Counterfactual}\\
                & Non-Denigrated & Denigrated  & Non-Denigrated & Denigrated   & Non-Denigrated & Denigrated & Non-Denigrated & Denigrated\\\midrule
          SS    &   \checkmark  & \checkmark &   \checkmark  &   \checkmark  &   \checkmark    &   \checkmark  &   \checkmark & \checkmark\\
          rSS   &   \#\refex{matchtomorrow}  & \checkmark\refex{acausalncfdenigrated}  & \checkmark\refex{match-repeat4}  &   \checkmark\refex{match-acausal-denigrated}  &   \#\refex{matchsnapnocf} & \checkmark\refex{causalncfdenigrated} &   \#\refex{matchsnapcf}    &   \checkmark\refex{match-causal-denigrated}\\
          \bottomrule
    \end{tabular}}\labtab{ourdata-repeat666}
\end{table}

Here, we have four factors to account for: (i) the obligatory need for contrastive stress in the antecedent that is required for felicity; (ii) that non-counterfactuality renders reverse Sobel sequences infelicitous; (iii) that a causal relation between $\phi$ and $\phi\land\psi$ causes reverse Sobel sequences to be infelicitous; and (iv) that the factor of counterfactuality and causality can be neutralised with the overt or contextual denigration of $\psi$ as a possibility (i.e., the questioning or denigration of $\psi$ may serve as a rescue operation to contrastively stressed reverse Sobel sequences).

In general terms, \textcite{Ippolito2020} accounts for the need of contrastive stress. In \citepos{Ippolito2020} account, the construction of the conditional question under discussion and the construction of the structured set of alternatives is dependent on the focus feature that \textcite{Ippolito2020} requires to be present in either antecedent. This focus feature is typically indicated via the stressing of some syntactic constituent. As the focus must be placed on the differing features in the antecedent to properly isolate the conditional question under discussion, \textcite{Ippolito2020} makes the same general prediction for the presence of (contrastive) stress in the antecedent as we do in \refsec{contrast-vss}. Where \textcite{Ippolito2020} and we differ is that we have posited the contrastive stressing of the antecedent's TAM morphology in the antecedent as a last resort possibility for stress placement (i.e., contrastively stressing the auxiliary verb in a counterfactual). However, \textcite{Ippolito2020} merely omits and does not contradict this possibility. As such, her account may be extended by our proposed semantics with no greater difficulty. We discuss the specifics of this in \refsec{awesome}, but concentrate on evaluating her unmodified proposal in this section.

As such, we may move on to the next factor: The factor of counterfactuality that renders (acausal) counterfactual reverse Sobel sequences possibly felicitous but non-counterfactual reverse Sobel sequences infelicitous, as demonstrated by \refex{matchtomorrow} and \refex{match-repeat4}, repeated below as \refex{doomsday-x-repeat666} and \refex{doomsday-y-repeat666}.
\ex\phantomsection\context{Concerning a dry match in a room with a large open source of water.}
    If I struck this match tomorrow and it was wet, it wouldn't light; \#but if I \MakeUppercase{were} to strike this match tomorrow, it would light.\labex{doomsday-x-repeat666}
\xe
\ex\phantomsection\context{Holding up a dry match, with no water around.}If I had struck this match and it had been soaked, it would not have lit. But if I \MakeUppercase{had} struck this match, it would have lit.\\%
\emptyfill(adapted from \textcite[p. 106]{Stalnaker1968} by \textcite[p. 487]{Lewis2018})\labex{doomsday-y-repeat666}
\xe
Unfortunately, \textcite{Ippolito2020} has explicitly restricted her analysis to counterfactual reverse Sobel sequences. As such, we cannot properly evaluate what her model would predict for non-counterfactual sequences. However, it should be noted that \textcite{Ippolito2020} entertained the notion of possibly extending her account to non-counterfactual conditionals. To this end, she presupposed that such an extension would only be possible if the analysis of counterfactuals and the analysis of non-counterfactuals are identical in the requirement of some kind of ordering, as her analysis rests on the use of similarity as a form of covert strengthening subject to economy. If indicatives did not have any similarity-ordering-like function that restricts their evaluation to some specific subset of the antecedent worlds, her framework could not be applied to them \parencite[p.~655,~Footnote~16]{Ippolito2020}. I.e., \citepos{Ippolito2020} framework requires a variably-strict indicative and subjunctive semantics for it to be applied to non-counterfactuals. In this regard, \textcite{Ippolito2020} makes a prediction on the semantics of non-counterfactuals opposite to ours in \refsec{showitworks}, where we posited that any non-counterfactual semantics must evaluate conditionals with respect to the entirety of possible antecedent worlds. As such, our approaches to (reverse) Sobel sequences would be incompatible with one another if \citepos{Ippolito2020} framework were to extend to non-counterfactual conditionals in the way envisioned by \textcite{Ippolito2020}.

As such, we would move on to the next factor: causality. As \textcite{Klecha2014} has observed, reverse Sobel sequences where $\phi$ precedes $\psi$ on some causal chain of events are infelicitous, whereas reverse Sobel sequences where no causal relation exists may be felicitous, as shown by the acausal sequence in \refex{doomsday-y-repeat666} above and by the causal sequence \refex{matchsnapcf}, repeated below in \refex{wonderwoman-repeat666}. 
\ex\phantomsection\context{Holding up a dry match (with no water around).}If I had struck this match and it had snapped, it would not have lit. \#But if I \MakeUppercase{had} struck this match, it would have lit.\labex{wonderwoman-repeat666}
\xe
Here it should be noted that \citepos{Ippolito2020} framework does not, on any fundamental level, differentiate between causal and acausal reverse Sobel sequences. Her explanation for the infelicity of causal reverse Sobel sequences is therefore either identical or very similar to her explanation on infelicitous acausal reverse Sobel sequences (depending on her views on how causality affects world similarity). The reason behind this is simple: Since \textcite{Ippolito2020} does not assume stress on the TAM morphology but places focus elsewhere in the antecedent, she never compares alternative domains with one another but individual propositions. As such, any additional specification $\psi$ to $\phi$ would put $\phi\land\psi$ in a subset relation to $\phi$, rendering $\phi\land\psi$ as, at least, a daughter node of $\phi$ according to \citepos{Ippolito2020} rules on how structured sets of alternatives are to be constructed. As such, the covert strengthening of world similarity would affect causal and acausal reverse Sobel sequences equally, violating the principle of economy in just the same fashion. Let us exemplify this with \refex{doomsday-y-repeat666} and \refex{wonderwoman-repeat666} with relocated focus, as shown in \refex{relocationfocus}:
\pex\phantomsection\labex{relocationfocus}\contextpex{Holding up a dry match (with no water around).}%
\a\phantomsection If I had [struck this match and it had been soaked]\textsubscript{F}, it would not have lit. But if I {had} [struck this match]\textsubscript{F}, it would have lit.
\a\phantomsection If I had [struck this match and it had snapped]\textsubscript{F}, it would not have lit. \#But if I had [struck this match]\textsubscript{F}, it would have lit.
\xe
Assuming, respectively, that each conditional shares the same conditional question under discussion, as previously elaborated, \textcite{Ippolito2020} would construct the following structured set of alternatives in \reffig{ippolito-matchsoak} for the acausal reverse Sobel sequence.
\begin{figure}[!htb]
    \centering
    \Tree [ [.{\textbf{strike match}} {\underline{strike match and it is wet}} {\textbf{strike match and it is dry}} ] {not strike match} ]
    \caption{The structured set of alternatives generated by \textit{[struck this match and it had been soaked]\textsubscript{F}} according to \textcite{Ippolito2020}. The original utterance is underlined, and all salient alternatives are in boldface.}
    \labfig{ippolito-matchsoak}
\end{figure}

For the acausal reverse Sobel sequence, the source of infelicitous is therefore abundantly clear: The $\phi$-conditional similarity function returns the closest $\phi$-worlds, which are, in turn, $\phi\land\neg\psi$-worlds due to the counterfactuality of $\psi$. As such, the overt $\phi$ is, by \citepos{Ippolito2020} standards, covertly strengthened to $\phi\land\neg\psi$, which has, in turn, as a sister node to $\phi\land\psi$ (sharing one entailment with $\phi\land\psi$), been raised to salience by the preceding $\phi\land\psi$-conditional, violating economic constraints.

For the causal reverse Sobel sequence, \textcite{Ippolito2020} would construct the structured set of alternatives in \reffig{ippolito-matchsnap}, which is identical in its hierarchy to \reffig{ippolito-matchsoak}.
\begin{figure}[!htb]
    \centering
    \resizebox{\linewidth}{!}{\Tree [ [.{\textbf{strike match}} {\underline{strike match and it snaps}} {\textbf{strike match and it does not snap}} ] {not strike match} ]}
    \caption{The structured set of alternatives generated by \textit{[struck this match and it had snapped]\textsubscript{F}} according to \textcite{Ippolito2020}. The original utterance is underlined, and all salient alternatives are in boldface.}
    \labfig{ippolito-matchsnap}
\end{figure}
For the causal reverse Sobel sequence, we have two differing ways on how infelicity may be derived, depending on how we assume causality impacts world similarity. Either we assume that the fact that $\phi$ precedes $\psi$ on some causal chain of events does not factor into the world similarity order or we follow \textcite{Bennett2003} and \citepos{Arregui2009} assumption that this would render $\phi$ and $\phi\land\psi$ equal in similarity to the evaluation world. If we go with the former approach, infelicity is derived in an identical fashion to the acausal reverse Sobel sequence: an economy-violating covert strengthening of $\phi$ to $\phi\land\neg\psi$ due to the nearest worlds being non-snapping worlds (since the match has not snapped in the evaluation world). If we go with \textcite{Bennett2003} and \citepos{Arregui2009} approach, the world similarity function would not strengthen $\phi$ to $\phi\land\neg\psi$, as the closest $\phi\land\neg\psi$-worlds and the closest $\phi\land\psi$-worlds would be equal in similarity to the evaluation world. As such, causal reverse Sobel sequences would not violate economy. However, they are still predicted to be infelicitous due to (i) making contradictory statements regarding the status of $\chi$ in some of the same worlds and (ii) them violating the specificity constraints proposed by \textcite{Ippolito2020}: \textit{strike match and it snaps} and \textit{strike match} are neither dominated by an equal number of nodes in the structured set of alternatives nor is either of them the only member of its branch. As such, the infelicity of causal reverse Sobel sequence is predicted either way.

It should be noted that going with \textcite{Bennett2003} and \textcite{Arregui2009} superficially appears to have the unfortunate consequence of the same issue applying to regularly ordered causal Sobel sequence: After all, the specificity constraint applies in either direction. As such, if $\phi$ was evaluated as the direct mother node of $\phi\land\psi$ from the start, it would lead to the same violation of the two conditionals' antecedents not being dominated by the same number of nodes in $T_{\mathbbm{A}_\textit{strike-match}}$. However, the solution to this issue lies in the nature of how the structured sets of alternatives represent the level of specificity. Following the reasoning of \textcite{Klecha2014,Klecha2015}, as explained and demonstrated in \refsec{CSS}, we may assume that a causal Sobel sequence's initial $\phi$-conditional is a form of loose talk, where the level of imprecision is high enough for the conditional to be evaluated as true enough. This lack of granularity, translated into \citepos{Ippolito2020} framework, would be equivalent to $T_{\mathbbm{A}_\text{strike-match}}$ not subdividing the $\phi$-node into $\phi\land\psi$ and $\phi\land\neg\psi$, as shown in \reffig{ippolito-matchsnap-less}.
\begin{figure}[!htb]
    \centering\hspace{-4cm}
    \Tree [ {\underline{strike match}} {\textbf{not strike match}} ]
    \caption{The imprecise structured set of alternatives generated by \textit{[struck this match]\textsubscript{F}} according to \citepos{Ippolito2020} model, assuming \textcite{Bennett2003} and \citepos{Arregui2009} world similarity metric. The original utterance is underlined, and all salient alternatives are in boldface.}
    \labfig{ippolito-matchsnap-less}
\end{figure}

\noindent This way, the specificity constraint, repeated as \refdef{specificity-repeat1}, does not come into effect.
\pex\phantomsection%
\extitle{Specificity Condition}\labdef{specificity-repeat1}
A sequence $\Sigma<[_{S_i}\ldots\alpha_F\ldots],[_{S_j}\ldots\beta_F\ldots]>$, s.t. both
$S_i$ and $S_j$ are answers to the same QUD and $\beta\in T_{\mathbbm{A}_\alpha}$, is felicitous if either \dots
\a\phantomsection \dots $\alpha$ or $\beta$ is the only node on its branch in $T_{\mathbbm{A}_\alpha}$, or \dots \labdef{specificity-klecharelevant-repeat1}
\a\phantomsection \dots $\alpha$ and $\beta$ are dominated by the same number of nodes in $T_{\mathbbm{A}_\alpha}$.\\\emptyfill\parencite[p.~643]{Ippolito2020}\labdef{specificity-sister-repeat1}
\xe
This is because the specificity constraint only applies to sequences that answer the same question under discussion and where the second item of the sequence is a member of the first item's structured set of alternatives, and the less granular $T_{\mathbbm{A}_\text{strike-match}}$ does not contain a node pertaining to such an alternative, as previously shown in \reffig{ippolito-matchsnap-less}. Only when the $\phi\land\psi$-conditional is uttered does the granularity increase and $\phi$ must be partitioned into disjoint sub-alternatives for the sake of evaluating the conditional which would ensure a violation of the specificity condition (this is functionally analogous to a formalised variant of \citepos{Klecha2014,Klecha2015} assumption of imprecision and precisification).

With this we may turn to our final factor: the overt or implicit denigration of the possibility of $\psi$. Incidentally, this is the only way a reverse Sobel sequence may be evaluated as felicitous in \citepos{Ippolito2020} current framework. When discussing the possibility of felicitous reverse Sobel sequences, \textcite[p.~663]{Ippolito2020} proposes that such sequences are only possible via eliminating the possibility of $\psi$---either overtly or via context. The reason for this is that we need to shift between two differing structured sets of alternatives. This is only possible, in the context of reverse Sobel sequences, if we re-partition logical space into a different number of disjoint alternatives compared to the previous structured set of alternatives. This, in turn, is only possible if we exclude the possibility of some worlds (which would be the $\psi$-worlds in case of Sobel sequences). \textcite{Ippolito2020} illustrated this process via \citepos{Moss2012} example shown in \refex{moss}.
\pex\phantomsection\labex{moss}\contextpex{Suppose John and Mary are our mutual friends. John was going to ask Mary to marry him, but chickened out at the last minute. I know Mary much better than you do, and you ask me whether Mary might have said yes if John had proposed. I tell you that I swore to Mary that I would never tell anyone that information, which means that strictly speaking, I cannot answer your question. But I say that I will go so far as to tell you two facts:}
\a\phantomsection If John had proposed to Mary and she had said yes, he would have been really happy.
\a\phantomsection But if John had proposed, he would have been really unhappy.\\\emptyfill\parencite[p. 577]{Moss2012}
\xe
Here, \textcite{Ippolito2020} argues that the $\phi\land\psi$-conditional constructs the following structured set of alternatives $T_{\mathbbm{A}_\text{propose-and-yes}}$ in \reffig{ippolito-marryyes}, where we allow for the possibility of Mary agreeing to John's proposal for the sake of evaluating whether or not the consequent would follow from the antecedent---despite knowing that Mary agreeing to it is actually impossible.
\begin{figure}[!htb]
    \centering
    \Tree [.{} [.{John proposes} {J proposes and M says yes} {J proposes and M says no} ] [.{John does not propose} ] ]
    \caption{The structured set of alternatives generated by \textit{[John proposed to Mary and she said yes]\textsubscript{F}} according to \textcite{Ippolito2020}.}
    \labfig{ippolito-marryyes}
\end{figure}

But, when we proceed to the evaluation of the $\phi$-conditional, knowing of the impossibility of $\psi$ and having no need to even entertain the possibility of such worlds, the speaker re-partitions the logical space so as to exclude the possibility of Mary saying yes to John's proposal. In other words, we would re-partition logical space into the two possibilities of \textit{John proposes and Mary says no} and \textit{John does not propose}, as shown by the structured set of alternatives in \reffig{ippolito-marryno}.
\begin{figure}[!htb]
    \centering\hspace{-4cm}
    \Tree [.{} [.{John proposes and Mary says no} ] [.{John does not propose} ] ]
    \caption{The structured set of alternatives generated by \textit{[John proposed to Mary]\textsubscript{F}} according to \textcite{Ippolito2020}.}
    \labfig{ippolito-marryno}
\end{figure}

This way, economy is not violated as $\phi$ is not strengthened to $\phi\land\neg\psi$ by competing against some $\phi\land\psi$-worlds in terms of similarity---the closest $\phi$-worlds simply happen to be $\phi\land\neg\psi$-worlds and there are no $\phi\land\psi$ in our world similarity ordering (since our similarity function is relativised to the set of alternatives, which no longer includes any alternative corresponding to $\phi\land\psi$). Whether or not this process is done implicitly, as in \refex{moss}, or explicitly, as in \refex{moss-explicit} as well as in the majority of our own examples for denigrated reverse Sobel sequences, does not matter from a formal point of view, as the process and results are perfectly identical.
\pex[nopreamble=true]\phantomsection\label{ex:moss-explicit}%
\a\phantomsection If John had proposed to Mary and she had said yes, he would have been really happy.
\a\phantomsection But if John had proposed, Mary would not have said yes.
\a\phantomsection So, if John had proposed, he would have been really unhappy.\\\emptyfill\parencite[p. 663]{Ippolito2020}
\xe

With this, we may summarise our findings concerning \citepos{Ippolito2020} basic and unmodified framework. We had four factors to account for: (i) contrastive stress, (ii) infelicity of non-counterfactuality, (iii) infelicity of causality, and (iv) felicity of $\psi$-denigrated reverse Sobel sequences. 

In reverse order, \textcite{Ippolito2020} completely accounts for (iv) the felicity of denigration (within the counterfactual limits of her framework). The stress is still required---despite denigration---due to the need to identify the conditional question under discussion for the construction of the structured sets of alternatives. Causal reverse Sobel sequences may be accounted for by re-partitioning the structured set of alternatives such that neither specificity nor economy is violated.

Concerning (iii), the complete infelicity of (non-denigrated) causal reverse Sobel sequences, \textcite{Ippolito2020} accounts for this by virtue of her account rendering all reverse Sobel sequences infelicitous short of denigration. As an unfortunate side-effect of this, there is no reason to expect---using \citepos{Ippolito2020} framework---that acausal reverse Sobel sequences are any likelier to be felicitous. Since \textcite{Ippolito2020} does not place stress on the auxiliary verb of counterfactuals, however, this is not altogether wrong, as counterfactuals who are not contrastively stressed via the auxiliary verb are stereotypically infelicitous (see \refch{SS} for details).

Concerning (ii), the infelicity of non-counterfactual reverse Sobel sequences, \textcite{Ippolito2020} has nothing explicit to say due to her framework being strictly limited to counterfactual conditionals. However, she noted that any direct translation of her framework into the realm of non-counterfactuality would require a variably-strict semantics where we only evaluate a subset of the indicative or future-less-vivid antecedent worlds. This prediction concerning the nature of non-counterfactual conditional semantics run directly counter to the predictions we ourselves have made in \refsec{showitworks}, where we showed that our model requires both indicative and future-less-vivid conditionals to quantify over all possible antecedent worlds of their respective domains (differentiating between an indicative and a future-less-vivid domain).

As such, we would tentatively conclude that our approach, outlined in \refch{pragmatics-SS}, has an advantage concerning the empirical coverage as well as accuracy in comparison to \citepos{Ippolito2020} unmodified account: Namely, \textcite{Ippolito2020} does not account for why contrastive stress typically falls upon the auxiliary verb if no other overtly contrastable lexical items exist. However, this is not an inherent error in her system but simply a piece of data that she did not have access to and therefore did not account for. As we show in the next section (i.e., \refsec{awesome}), her account can easily be extended to account for this phenomenon as well by adopting some of our own assumptions. As such, both of our accounts would be on par with one another with regards to retrodicting all of the known data (as we show in the next section). However, there are some components that we entirely disagree with. Let us consider the purpose of economic constraint as described in \refdef{economy}, repeated below as \refdef{economy-satan}.
\ex\phantomsection\labdef{economy-satan}
\extitle{Economy}
For any sequence $\Sigma=<[_{S_{n-1}}\ldots\alpha_F\ldots],[_{S_{n}}\ldots cs(\beta_F)\ldots]>$, where $cs$ is a covert strengthening operator and $\beta\in T_{\mathbbm{A}_\alpha}$:\\
$\#\Sigma$ if $\exists\gamma\in\Delta_\alpha$ s.t. the sentence $S'_n$ obtained by replacing $cs(\beta)$ in $S_n$ with $\gamma$ is equivalent to $S_n$.
\xe
In linguistics, we typically place constraints upon voluntary mechanisms on basis of their relative cost to other possible alternatives. In the context of implicatures that means that we do not generate them if doing so would incur too great a cost (e.g., when their overt counterparts are already made salient due to previous discourse). This makes sense as implicatures are, by and large, a non-compulsory cancellable process. As such, it makes sense that not generating them at all---even at the cost of infelicity, as seen with Gazdar's disjuncts in \refsec{ippolito-disjunction}---is an option in the face of easier alternatives. The issue here is that \textcite{Ippolito2020} is extending these economic constraints to an obligatory hard-coded part of the grammar: the similarity function of conditionals. It seems unlikely to us that an obligatory process such as this is subject to the very same constraints as voluntary processes. After all, if something is an obligatory process, you would, by definition, still be compelled to commit to it, regardless of the costs that may occur (as you cannot simply exclude the similarity function from conditional semantics).

\section{Merging Our Account With Ippolito (2020)}\labsec{awesome}
Nevertheless, setting aside the debatable issue of economy, we attempt to incorporate our own findings and semantics regarding contrastively stressed TAM morphology into \citepos{Ippolito2020} framework in this section, in an attempt to improve the coverage of \citepos{Ippolito2020} account. Conversely, this can also be seen as further formalising the process of imprecision and precisification which our proposal has inherited from \textcite{Klecha2014,Klecha2015} along the lines of \citepos{Ippolito2020} more specific and formally more complex system of specificity.

First, let us briefly review the semantics of contrastively stressed TAM morphology in antecedents as detailed in \refsec{contrast-vss}. We assume that TAM morphology is connected to the properties of the world variable \parencites{Palmer1986}{Iatridou2000}{Arregui2009}{Romero2014}[amongst others]{Schulz2014} such that the stress placed upon the auxiliary verb carrying the TAM information actually targets the quantificational domain that is derived by the antecedent in relation to its consequent. We modelled this along the lines of \citepos{Jacobson2000} account for contrastively stressed pronouns, where the pronouns were rendered as partial identity functions restricted to the domain of its binder, where the actual contrast was accomplished by comparing the two necessarily disjoint identity functions. As such, 
the LF of a reverse Sobel sequence would correspond to the form given in \refex{identityw-variably-strict}, repeated below as \refex{identityw-variably-strict-repeat666}.
\pex[nopreamble=true]\phantomsection\label{ex:identityw-variably-strict-repeat666}%
\a\phantomsection If $[\lambda w_s. \phi(\text{\textbf{\color{black}\scshape id\textsubscript{closest-$\phi\land\psi$}}}(w))\land \psi(\text{\color{black}\scshape id\textsubscript{closest-$\phi\land\psi$}}(w))]$, (then) $[\lambda w_s.\neg\chi(w)]$.
\a\phantomsection If $[\lambda w_s. \phi(\text{\textbf{\color{black}\scshape id\textsubscript{closest-$\phi$}}}(w))]$, (then) $[\lambda w_s.\chi(w)]$.
\xe
The formal implementation of which was shown in \refex{identityw-variably-strict-demo}, where the accessibility function $f_\leqslant(p,w)$ returns the set of $p$-worlds closest to the evaluation world $w$, as repeated below in \refex{identityw-variably-strict-demo-repeat666}.
\pex[nopreamble=true]\phantomsection\label{ex:identityw-variably-strict-demo-repeat666}%
\a\phantomsection $\intension{If $\phi$ and $\psi$, not $\chi$}=[\lambda w_s.\forall v:v\in f_\leqslant([\lambda w'_s.\phi(\text{\textbf{\color{black}\scshape id\textsubscript{closest-$\phi\land\psi$}}}(w'))$\\\emptyfill$\land\psi(\text{\textbf{\color{black}\scshape id\textsubscript{closest-$\phi\land\psi$}}}(w'))],w)[\neg\chi(v)]$
\a\phantomsection $\intension{If $\phi$, $\chi$}=[\lambda w_s.\forall v:v\in f_\leqslant([\lambda w'_s.\phi(\text{\textbf{\color{black}\scshape id\textsubscript{closest-$\phi$}}}(w'))],w)[\chi(v)]$
\xe
The contrast between the two identity functions was deemed to be successful iff the two domains covered by them were entirely disjoint. In addition to this, we had two further important assumptions: First, we adopt \textcite{Bennett2003} and \citepos{Arregui2009} assumption that a causal relation between two propositions $\phi$ and $\psi$ prevents a change in world similarity from $\phi$ to $\phi\land\psi$. Second, we mandated the assumption that non-counterfactual conditionals quantify over domains that are not further subdivided into different levels of world closeness, but that indicative and future-less-vivid worlds each represent a single degree of world similarity.

How would our proposed model interact with \citepos{Ippolito2020} system? In order to answer this question, we must first consider what alternatives are generated by putting focus on the identity function. To this end, we would propose that the alternatives generated by {\scshape id\textsubscript{closest-$\phi$}} correspond to identity functions whose domains represent different levels of world similarity and granularity. As such, we would argue that the focus value of {\scshape id\textsubscript{closest-$\phi$}} corresponds to \refex{alternativeids}.
\ex\phantomsection\label{ex:alternativeids}
$\intension[f]{{\scshape id\textsubscript{closest-$\phi$}}}=\{\intension[]{{\scshape id\textsubscript{closest-$\phi$}}},\intension[]{{\scshape id\textsubscript{closest-$\phi\land\psi$}}},\intension[]{{\scshape id\textsubscript{closest-$\phi\land\psi'$}}},\ldots\}$
\xe
In order to implement {{\scshape id\textsubscript{closest-$\phi$}}} into \citepos{Ippolito2020} system, these alternatives then construct a structured set of alternatives $T_{\mathbbm{A}_\text{{\scshape id\textsubscript{closest-$\phi$}}}}$ such that (i) the denotation of each daughter node is a proper subset of the denotation of its mother node (condition of strength), (ii) the denotation of each alternative is disjoint to the denotations of all of its sister nodes (condition of disjointness), and (iii) the union of the denotations of all sister nodes are equal to the denotation of their mother node (condition of exhaustivity), as specified in the construction rules for structured sets of alternatives in \refdef{alternativeconstruction}, and as repeated below in \refdef{alternativeconstruction-repeat1}.
\pex\phantomsection\label{def:alternativeconstruction-repeat1}
\pextitle{Well-Formedness Conditions for Structured Sets of Alternatives}\\
$T_\mathbbm{A}$ is well-formed iff all of the following conditions are met:
\a\phantomsection \pextitle{Strength:} for any two alternatives $\alpha,\beta\in\mathbbm{A}$, $\beta$ is the daughter of $\alpha$ in $T_\mathbbm{A}$ just in case $\intension[]{$\beta$}\subset\intension[]{$\alpha$}$.
\a\phantomsection \pextitle{Disjointness:} for any two alternatives $\beta_1,\beta_2\in\mathbbm{A}$, if $\beta_1$ and $\beta_2$ are sisters in $T_\mathbbm{A}$, then $\intension[]{$\beta_1$}\cap\intension[]{$\beta_2$}=\emptyset$.
\a\phantomsection \pextitle{Exhaustivity:} for any alternative $\alpha$ with daughters $\beta_1,\ldots,\beta_n$ in $T_\mathbbm{A}$, $\intension[]{$\beta_1$}\cup\ldots\cup\intension[]{$\beta_n$}=\intension[]{$\alpha$}$.\hfill\parencite[p. 640]{Ippolito2020}
\xe

As such, due to the nature of similarity, we can establish the following rules for the construction of the structured set of alternatives: (i) The supreme mother node would be the evaluation of the antecedent with respect to all $\phi$-worlds; (ii) since any decrease in world similarity enforces disjoint sets in relation to all other possible values of world similarity, each degree of similarity is represented as a daughter node of the supreme mother node, cumulatively covering all $\phi$-worlds, thereby satisfying the conditions of strength, disjointness, and exhaustivity; (iii) since there is no decrease in world similarity for $\phi\land\psi'$ if $\phi$ causally precedes $\psi'$ on some causal chain events or $\phi$ is non-counterfactual by nature, causal or non-counterfactual $\phi\land\psi'$ domains both exhaustively partition the mother node of their world similarity value with a suitable complement domain. This way, we arrive at the structured set of alternatives $T_{\mathbbm{A}_\text{{\scshape id\textsubscript{closest-$\phi$}}}}$ in \reffig{ippolito-awesome}.
\begin{figure}[!htb]
    \centering
    \resizebox{\textwidth}{!}{\Tree [.{{\scshape id\textsubscript{$\phi$}}} [.{{\scshape id\textsubscript{closest-$\phi$}}} {{\scshape id\textsubscript{closest-$\phi\land\psi_3$}}} {{\scshape id\textsubscript{closest-$\phi\land\neg\psi_3$}}} ] [.{{\scshape id\textsubscript{closest-$\phi\land\psi_1$}}} {\dots} {\dots} ] [.{{\scshape id\textsubscript{closest-$\phi\land\psi_2$}}} {\dots} {\dots} ] {\dots} ]}
    \caption{The proposed structured set of alternatives generated by focus on TAM morphology in the antecedent within \citepos{Ippolito2020} framework, where $\psi_1,\psi_2$ represent some counterfactual propositions with no causal link to $\phi$, and where $\psi_3$ represents some proposition that is causally linked to $\phi$ or is non-counterfactual by nature.}
    \labfig{ippolito-awesome}
\end{figure}

Given this structured set of alternatives, we have two possible avenues for further implementation. The first avenue is a parsimonious implementation of \citepos{Ippolito2020} account. Here, we would assume that the world identity function behaves just like any other focused item when it comes to covert strengthening: i.e., a $\phi$-conditional is covertly strengthened from $\predicate{id}_\phi$ to $\predicate{id}_{\text{\scshape closest-}\phi}$ and, likewise, a $\phi\land\psi$-conditional is covertly strengthened from $\predicate{id}_{\phi\land\psi}$ to $\predicate{id}_{\text{\scshape closest-}\phi\land\psi}$. This would cause obvious issues: As can be seen in \reffig{ippolito-awesome}, the utterance of the $\phi$-conditional would raise its sibling nodes in $T_{\mathbbm{A}_\text{{\scshape id\textsubscript{closest-$\phi$}}}}$ to salience, which would include $\predicate{id}_{\text{\scshape closest-}\phi\land\psi}$. As such, covertly strengthening anything to $\predicate{id}_{\text{\scshape closest-}\phi\land\psi}$ would violate the principle of economy---and any sensible alterations to the structured set of alternatives to prevent the raising to salience would entail a violation of the specificity constraint.\footnote{One such alternative structure of alternatives would be to have an intermediary $\predicate{id}_{\phi\land\psi}$ node such that it is a sibling node to $\predicate{id}_{\text{\scshape closest-}\phi}$ and a mother node to $\predicate{id}_{\text{\scshape closest-}\phi\land\psi}$. This would, however, entail that $\predicate{id}_{\text{\scshape closest-}\phi\land\psi}$ and $\predicate{id}_{\text{\scshape closest-}\phi}$ are of different levels of specificity, ensuring that any (reverse) Sobel sequence would violate \citepos{Ippolito2020} specificity constraint.} This would predict that regularly ordered Sobel sequences are universally infelicitous, which is obviously contrary to fact. As such, our proposed model from \refch{pragmatics-SS} would be incompatible with this parsimonious adaptation of \textcite{Ippolito2020}. The second avenue is a less parsimonious implementation of \textcite{Ippolito2020}. Here, we would assume that the covert strengthening takes place prior to the pragmatic determination of the range of the antecedent's world identity function. This way, the $\phi$-conditional would start off at $\predicate{id}_{\text{\scshape closest-}\phi}$ rather than $\predicate{id}_{\phi}$. Likewise, the $\phi\land\psi$-conditional would start off at $\predicate{id}_{\text{\scshape closest-}\phi\land\psi}$ rather than $\predicate{id}_{\phi\land\psi}$. Given this assumption, it is not possible to violate the principle of economy, as it is impossible that the overt form covertly deviates from its node in the structured set of alternatives---as it already represents the end result of the covert strengthening in the antecedent. Therefore, the only possible violations for (reverse) Sobel sequences are ones of specificity. In the context of our structured set of alternatives, this would entail that any vertical movements introduced via causality or non-counterfactuality immediately makes a reverse Sobel sequence violate the specificity constraint, rendering it infelicitous. The only possible discourse moves that do not result in a violation of this constraint are moves where $\psi$ represents a counterfactual proposition that is causally independent of $\phi$ such that $\phi\land\psi$ represents a lateral movement in the structured set of alternatives. This is due to how \citepos{Ippolito2020} system would construct our set of alternatives, as shown in \reffig{ippolito-awesome}, given our assumptions regarding world similarity from \refch{pragmatics-SS}. We cover this in a more detailed fashion in the next section (i.e., \refsec{showitworks-withippolito}).

\subsection{Retrodiction: Accounting for All Available Data With Specificity}\labsec{showitworks-withippolito}
We inherit \citepos{Ippolito2020} reasoning for reverse Sobel sequences where the contrastive stress is not placed upon the TAM-morphology of the antecedent. We do not repeat that part of her account here, but refer to \refsec{ippolito-rss} for details. We now focus exclusively on how \citepos{Ippolito2020} framework merged with our own assumptions and how it can derive all of the known empirical data pertaining to reverse Sobel sequence where the contrastive stress falls upon TAM morphology. 

Incorporating the semantics of contrastively stressed TAM morphology as described in the previous section, \citepos{Ippolito2020} framework is capable of deriving all of the required empirical observations. We had four factors to account for: (i) contrastive stress, (ii) infelicity of non-counterfactuality, (iii) infelicity of causality, and (iv) felicity of $\psi$-denigrated reverse Sobel sequences. The distribution of felicity for Sobel sequences and reverse Sobel sequences was shown in \reftab{ourdata}, repeated below as \reftab{ourdata-repeat999}.
\begin{table}[!htb]
\caption{Current empirical data on felicity distribution, broken down by causality, counterfactuality, and overt denigration of relevance (or implicit epistemic exclusion) of $\psi$, with example numbers that exemplify each reverse Sobel sequence condition. Contrastive stress on the auxiliary verb is assumed for all reverse Sobel sequences.}
\resizebox{\textwidth}{!}{
    \begin{tabular}{lcccccccc}\toprule
                &   \multicolumn{4}{c}{Acausal}     &  \multicolumn{4}{c}{Causal}\\
                & \multicolumn{2}{c}{Non-Counterfactual}  &   \multicolumn{2}{c}{Counterfactual}    & \multicolumn{2}{c}{Non-Counterfactual}  &   \multicolumn{2}{c}{Counterfactual}\\
                & Non-Denigrated & Denigrated  & Non-Denigrated & Denigrated   & Non-Denigrated & Denigrated & Non-Denigrated & Denigrated\\\midrule
          SS    &   \checkmark  & \checkmark &   \checkmark  &   \checkmark  &   \checkmark    &   \checkmark  &   \checkmark & \checkmark\\
          rSS   &   \#\refex{matchtomorrow}  & \checkmark\refex{acausalncfdenigrated}  & \checkmark\refex{match-repeat4}  &   \checkmark\refex{match-acausal-denigrated}  &   \#\refex{matchsnapnocf} & \checkmark\refex{causalncfdenigrated} &   \#\refex{matchsnapcf}    &   \checkmark\refex{match-causal-denigrated}\\
          \bottomrule
    \end{tabular}}\labtab{ourdata-repeat999}
\end{table}

\noindent First, we account for reverse Sobel sequences in \refsec{ourippolito-rss}. Then we account for regularly ordered Sobel sequences in \refsec{ourippolito-ss}.

\subsubsection{Reverse Sobel Sequences}\labsec{ourippolito-rss}
Regarding contrastive stress, not only does our combined framework account for obligatory stress on TAM morphology in absence of other viable constituents, it also retains \citepos{Ippolito2020} independent motivation for the need for stress---i.e., we need stress in the antecedent of conditionals in order to construct the conditional question under discussion, which, in turn, is required to construct the set of alternatives. Furthermore, the way \textcite{Ippolito2020} constructs structured sets of alternatives gives another independent motivation for why the partial identity functions must be disjoint in domains, as all nodes in her structured sets of alternatives must be exhaustively and disjointly partitioned amongst its daughter nodes.

Regarding non-counterfactuality, we are capable of explaining why reverse Sobel sequences such as \refex{matchtomorrow} and \refex{match-repeat4}, repeated below as \refex{doomsday-z666} and \refex{doomsday-zz666}, are infelicitous.
\ex\phantomsection\context{Concerning a dry match in a room with a large open source of water.}
    If I struck this match tomorrow and it was wet, it wouldn't light; \#but if I \MakeUppercase{were} to strike this match tomorrow, it would light.\labex{doomsday-z666}
\xe
\ex\phantomsection\context{Holding up a dry match, with no water around.}If I had struck this match and it had been soaked, it would not have lit. But if I \MakeUppercase{had} struck this match, it would have lit.\\%
\emptyfill(adapted from \textcite[p. 106]{Stalnaker1968} by \textcite[p. 487]{Lewis2018})\labex{doomsday-zz666}
\xe
\begin{figure}[!htb]
    \centering
    \resizebox{\textwidth}{!}{\Tree [.{{\scshape id\textsubscript{$\phi$}}} [.{\textbf{\scshape id\textsubscript{closest-$\boldsymbol{\phi}$}}} {\underline{\scshape id\textsubscript{closest-$\phi\land\psi_3$}}} {\textbf{\scshape id\textsubscript{closest-$\boldsymbol{\phi\land\neg\psi_3}$}}} ] [.{{\scshape id\textsubscript{closest-$\phi\land\psi_1$}}} {\dots} {\dots} ] [.{{\scshape id\textsubscript{closest-$\phi\land\psi_2$}}} {\dots} {\dots} ] {\dots} ]}
    \caption{The proposed structured set of alternatives generated by focus on TAM morphology in the antecedent of the $\phi\land\psi$-conditional of either a causal reverse Sobel sequence or a non-counterfactual reverse Sobel sequence, where $\psi_1,\psi_2$ represent some counterfactual propositions with no causal link to $\phi$, and where $\psi_3$ represents some proposition that is causally linked to $\phi$ or is non-counterfactual by nature.}
    \labfig{ippolito-awesome-causalornoncf}
\end{figure}
As we consider them to quantify over a single domain for indicatives and future-less-vivids each, any specification of $\phi$ via some $\psi$ introduces a vertical move between $\phi$ and $\phi\land\psi$ in the hierarchy of our structured alternative set. In \reffig{ippolito-awesome-causalornoncf}, we see that the utterance of a non-counterfactual reverse Sobel sequence's $\phi\land\psi$-conditional quantifies over a subdomain of the closest $\phi$-worlds---the closest $\phi\land\psi_3$ worlds. The subsequent $\phi$-conditional would quantify over the former conditionals mother node---all closest $\phi$-worlds---since $\phi$ would not be strengthened to $\phi\land\neg\psi_3$ as the closest $\phi\land\neg\psi_3$-worlds are equally close to the evaluation world as the closest $\phi$-worlds and the closest $\phi\land\psi_3$-worlds. This automatically violates the constraint of specificity, rendering such reverse Sobel sequences infelicitous.\footnote{Naturally, this reasoning would also extend to regularly ordered Sobel sequences, predicting non-counterfactual Sobel sequences to be as infelicitous as their reverse counterparts. We explain how this issue is circumvented in \refsec{ourippolito-ss}.} 

Regarding causality, we are also capable of accounting for their universal infelicity which was demonstrated with \refex{matchsnapcf}, repeated below as \refex{wonderwoman-repeat999}.
\ex\phantomsection\context{Holding up a dry match (with no water around).}If I had struck this match and it had snapped, it would not have lit. \#But if I \MakeUppercase{had} struck this match, it would have lit.\labex{wonderwoman-repeat999}
\xe
As any proposition $\psi$ that is causally preceded by $\phi$ on some causal chain of events is formally treated identical to how non-counterfactuals are treated. We would derive the same situation illustrated in \reffig{ippolito-awesome-causalornoncf}. As causal reverse Sobel sequences therefore involve a vertical movement in the structured set of alternatives, we automatically violate the constraint of specificity.\footnote{Naturally, this reasoning would also extend to regularly ordered Sobel sequences, predicting causal Sobel sequences to be as infelicitous as their reverse counterparts, same as with non-counterfactual Sobel sequences. We explain how this issue is circumvented in \refsec{ourippolito-ss} with a single explanation for both Sobel sequence variants.} 

With this, we come to the universal felicity of acausal counterfactual reverse Sobel sequences, demonstrated with \refex{match-repeat4}, repeated below as \refex{greenlantern-repeat999}.
\ex\phantomsection\context{Holding up a dry match, with no water around.}If I had struck this match and it had been soaked, it would not have lit. But if I \MakeUppercase{had} struck this match, it would have lit.\\%
\emptyfill(adapted from \textcite[p. 106]{Stalnaker1968} by \textcite[p. 487]{Lewis2018})\labex{greenlantern-repeat999}
\xe
Assuming the structured set of alternatives generated in \reffig{ippolito-awesome}, where each level of similarity is a separate daughter node to the set's supreme mother node, an acausal counterfactual reverse Sobel sequence would obligatorily quantify over sibling nodes. This is shown in \reffig{ippolito-awesome-acausal}.
\begin{figure}[!htb]
    \centering
    \resizebox{\textwidth}{!}{\Tree [.{\textbf{\scshape id\textsubscript{$\boldsymbol{\phi}$}}} [.{\textbf{\scshape id\textsubscript{closest-$\boldsymbol{\phi}$}}} {{\scshape id\textsubscript{closest-$\phi\land\psi_3$}}} {{\scshape id\textsubscript{closest-$\phi\land\neg\psi_3$}}} ] [.{\underline{\scshape id\textsubscript{closest-$\phi\land\psi_1$}}} {\dots} {\dots} ] [.{\textbf{\scshape id\textsubscript{closest-$\boldsymbol{\phi\land\psi_2}$}}} {\dots} {\dots} ] {\textbf{\dots}} ]}
    \caption{The proposed structured set of alternatives generated by focus on TAM morphology in the antecedent of the $\phi\land\psi$-conditional of an acausal reverse Sobel sequence, where $\psi_1,\psi_2$ represent some counterfactual propositions with no causal link to $\phi$, and where $\psi_3$ represents some proposition that is causally linked to $\phi$ or is non-counterfactual by nature.}
    \labfig{ippolito-awesome-acausal}
\end{figure}
Here, the $\phi\land\psi$-conditional would quantify over the closest $\phi\land\psi_1$-worlds, and the $\phi$-conditional would quantify over the closest $\phi$-worlds; i.e., a sister node to the closest $\phi\land\psi_1$-worlds. As such, the principle of specificity would be adhered to, ensuring that acausal reverse Sobel sequences are universally predicted to be felicitous.

Finally, we examine the rescue operation of covert and overt denigration, which guarantees felicity regardless of the factors of causality and counterfactuality, as shown in \refex{acausalncfdenigrated}, \refex{causalncfdenigrated}, \refex{match-acausal-denigrated}, and \refex{match-causal-denigrated}, which are repeated below as \refex{superman1-repeat999}, \refex{superman2-repeat999}, \refex{superman3-repeat999}, and \refex{superman4-repeat999}, respectively.
\pex[nopreamble=true]\phantomsection%
\a\phantomsection\context{Concerning a dry match in a room with a large open source of water.}
    If I struck this match tomorrow and it was wet, it wouldn't light. But there is little chance of this match becoming wet; so, if I \MakeUppercase{were} to strike this match tomorrow, it would light.\labex{superman1-repeat999}
\a\phantomsection\context{Holding up a dry match (with no water around).}If I struck this match and it snapped, it would not light. But the chances of me snapping a match are really, really low; so, if I \MakeUppercase{were} to strike this match, it would light.\labex{superman2-repeat999}
\a\phantomsection\context{Holding up a dry match, with no water around}If I had struck this match and it had been soaked, it would not have lit. But, as we know, this match is dry, so if I \MakeUppercase{had} struck this match, it would have lit.\labex{superman3-repeat999}
\a\phantomsection\context{Holding up a dry match (with no water around).}If I had struck this match and it had snapped, it wouldn't have lit. But the chances of the match breaking would've been very, very, \MakeUppercase{very} low, since I know what I'm doing. So, if I \MakeUppercase{had} struck this match, it would have lit.\labex{superman4-repeat999}
\xe
Here, we retain \citepos{Ippolito2020} reasoning from the previous section, and therewith an explanation for their universal felicity: Implicit or explicit denigration of $\psi$ causes a restructuring of the structured set of alternatives such that all instances pertaining to $\psi=1$ are eliminated, which disables the infelicity-deriving mechanisms of \citepos{Ippolito2020} framework. This way, the structured set of alternatives in \reffig{ippolito-awesome} would be restructured to \reffig{ippolito-awesome-denigration}, where the possibility of $\psi_3$ was denigrated for illustrative purposes, explaining the universal felicity of denigrated sequences.
\begin{figure}[!htb]
    \centering
    \Tree [.{{\scshape id\textsubscript{$\phi$}}} {{\scshape id\textsubscript{closest-$\phi\land\neg\psi_3$}}}  [.{{\scshape id\textsubscript{closest-$\phi\land\psi_1\land\neg\psi_3$}}} {\dots} {\dots} ] [.{{\scshape id\textsubscript{closest-$\phi\land\psi_2\land\neg\psi_3$}}} {\dots} {\dots} ] {\dots} ]
    \caption{The proposed structured set of alternatives generated by focus on TAM morphology in the antecedent within \citepos{Ippolito2020} framework, where $\psi_1,\psi_2$ represent some counterfactual propositions with no causal link to $\phi$, and where $\psi_3$, which represented some proposition that is causally linked to $\phi$ or is non-counterfactual by nature, was denigrated.}
    \labfig{ippolito-awesome-denigration}
\end{figure}

With this, \citepos{Ippolito2020} framework is capable of deriving the entire felicity distribution of reverse Sobel sequences correctly.

\subsubsection{Regularly Ordered Sobel Sequences}\labsec{ourippolito-ss}
For regularly ordered Sobel sequences, we need to account for two general cases: The felicity of acausal counterfactual Sobel sequences and the felicity of all other Sobel sequences (i.e., acausal ones and non-counterfactual ones).

For acausal counterfactual Sobel sequences, the felicity is explained in an identical fashion to the felicity of reverse Sobel sequences. As the only discourse moves available to us in such situations are lateral in the structured set of alternatives, their felicity is guaranteed. We refer to the previous section (i.e., \refsec{ourippolito-rss}) for details.

For causal and non-counterfactual Sobel sequences, we require a slightly more complicated story---a story, however, that is functionally equivalent to how we accounted for the felicity of causal Sobel sequences using \textcite{Bennett2003} and \textcite{Arregui2009} in \refsec{ippolito-comparison}. The issue here is that, given the structured set of alternatives in \reffig{ippolito-awesome}, a causal or non-counterfactual Sobel sequence would violate the principle of specificity in the same way its reverse counterpart does: One conditional quantifies over a daughter node of the other conditional. To prevent this, we would propose that---in the spirit of \citepos{Klecha2014,Klecha2015} proposal of imprecision---the $\phi$-conditional does not actually generate the structured set of alternatives in \reffig{ippolito-awesome}, but a less granular version of it, where the daughter nodes of the supreme mother node are not subdivided into further subdomains. This is shown in the upper half of \reffig{ippolito-awesome-imprecise}.
\begin{figure}[!htb]
    \centering
    \Tree [.{\textbf{\scshape id\textsubscript{$\boldsymbol{\phi}$}}} {\underline{\scshape id\textsubscript{closest-$\phi$}}}  {\textbf{\scshape id\textsubscript{closest-$\boldsymbol{\phi\land\psi_1}$}}} {\textbf{\scshape id\textsubscript{closest-$\boldsymbol{\phi\land\psi_2}$}}}  {\textbf{\dots}} ]

\hbox{}\vspace{2.5mm}
\hspace{55mm}\begin{tikzpicture}
\coordinate (O) at (0,0);
\draw [-{Stealth[scale=1]},line width=0.8mm](0,-0.5) -- (0,-2);
\node at (2,-1.25) {\textbf{\scshape Restructure}};
\end{tikzpicture}

\resizebox{\linewidth}{!}{\Tree [.{{\scshape id\textsubscript{$\phi$}}} [.{\textbf{\scshape id\textsubscript{closest-$\boldsymbol{\phi}$}}} {\underline{\scshape id\textsubscript{closest-$\phi\land\psi_3$}}} {\textbf{\scshape id\textsubscript{closest-$\boldsymbol{\phi\land\neg\psi_3}$}}} ] [.{{\scshape id\textsubscript{closest-$\phi\land\psi_1$}}} {\dots} {\dots} ] [.{{\scshape id\textsubscript{closest-$\phi\land\psi_2$}}} {\dots} {\dots} ] {\dots} ]}
    \caption{The proposed structured set of alternatives generated by focus on TAM morphology in the antecedent of an imprecise $\phi$-conditional within \citepos{Ippolito2020} framework, where $\psi_1,\psi_2$ represent some counterfactual propositions with no causal link to $\phi$, and where all further non-counterfactual or causal subdivisions of each daughter node is omitted. This is followed by a precisification-based restructuring of said structured set of alternatives where the non-counterfactual or causal subdivisions of each daughter node are represented.}
    \labfig{ippolito-awesome-imprecise}
\end{figure}
Only when the $\phi\land\psi$-conditional must be evaluated is the granularity of the structure increased such that it becomes equivalent to the structure in \reffig{ippolito-awesome}, as shown in the lower half of \reffig{ippolito-awesome-imprecise}. This way, the node that represents the $\phi\land\psi$-conditional is not part of the structure generated by the $\phi$-conditional. This renders the Sobel sequence non-subject to the specificity condition, explaining their felicity. Furthermore, this system---which is functionally equivalent to \citepos{Klecha2014,Klecha2015} imprecision and precisification model---also accounts for the imprecision-based flavour of causal and non-counterfactual reverse Sobel sequences that were respectively observed by \textcite{Klecha2014,Klecha2015} and by us in \refsec{showitworks-ss}.

\section{Intermediate Conclusion}
With this, we can account for the entire (in-)felicity distribution concerning reverse Sobel sequences and regularly ordered Sobel sequences with a variably-strict semantics where we make use of \citepos{Ippolito2020} framework to formalise the process of imprecision and precisification. In an improvement over our account from \refch{pragmatics-SS}, we can---in addition to deriving the need for contrastive stress, the appropriate felicity distribution, and the flavour of imprecision for acausal and non-counterfactual sequences---also provide an independent motivation and explanation from \citepos{Ippolito2020} framework for why the contrasting domains must be disjoint.
