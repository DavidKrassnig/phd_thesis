\chapter{Conclusion}\labch{conclusion}
The overall aim of this thesis was to contribute to the debate on whether or not conditionals should be modelled in a variably-strict or in a (semi-)dynamic strict manner. We did this by examining two phenomena that provide a window into what is required of an accurate conditional semantics: We examined the issue of negative polarity items in \refch{npi} and \refch{npi-conditionals} as well as the issue of (reverse) Sobel sequences in \refch{SS}, \refch{pragmatics-SS}, and \refch{ippolito}.

In \refch{npi}, we examined the two main approaches to NPI licensing: The traditional environment-/monotonicity-based approach to NPI licensing \parencite{Fauconnier1975a,Fauconnier1975b,Ladusaw1980,Giannakidou1998,Fintel1999} as well as the operator-based approach to NPI licensing that posits that NPIs are licensed by an \textit{even}-like presuppositional particle \parencite{Lee1994,Lahiri1998,Crnic2011,Crnic2014-dogma,Crnic2014-nm,Jeong2021}. We have shown that the former is unable to account for the contextually changing felicity of NPIs in non-monotone constructions \parencite{Crnic2014-nm}. We then examined the \textit{even}-based licensing theory, showing that it possess a higher accuracy in predicting the felicity status of NPI constructions. We more closely examined the negative bias that over \textit{even} and focused weak NPIs induce in polar questions, evaluating some of the proposed solutions by \textcite{Crnic2014-dogma,Crnic2014-nm} and \textcite{Jeong2020,Jeong2021}. We showed that either proposed solution, by itself, is unable to account for both the lack of bias of unfocused NPI questions and the negative bias of focused NPI questions at the same time. We then showed that the negative bias as well as the difference in negative bias can be accounted for by selectively merging both accounts together. By discarding the inquisitive semantic framework of \textcite{Jeong2020,Jeong2021} but retaining their proposed focus-triggered additive inference of {\scshape even} in combination with a question semantics along the lines of \textcite{Guerzoni2014-enviro}, we can uniformly explain not only why overt \textit{even ONE} and focused NPI questions exhibit differing degrees of negative bias but why unfocused NPI questions do not exhibit any negative bias by themselves whatsoever.

In \refch{npi-conditionals}, having had improved upon the operator-based approach in the previous chapter, we then applied the account to the two competing conditional semantic frameworks: the variably-strict and the (semi-)dynamic strict account. There, we showed that the (semi-)dynamic strict conditional semantics yields more accurate predictions for NPI felicity than the regular static variably-strict conditional semantics. We also showed that this disparity can be ameliorated to a very high degree by using a dynamic semantics in conjunction with a variably-strict semantics along the lines of \parencite{vanRooij2006} and \parencite{Walker2015}. However, the (semi-)dynamic strict approach retains a slight edge over the variably-strict approach even under those circumstances. Ultimately, however, we postulated that the variably-strict account may be further modified, though we were unable to do so ourselves, to also account for this last remaining discrepancy.

In \refch{SS}, we examined the empirical felicity distribution of reverse Sobel sequences. We examined the model proposed by \textcite{Klecha2014,Klecha2015} that tries to separate Sobel sequences into two independent phenomena with identical surface structures: the True Sobel sequences and the Lewis sequences, which we refer to as acausal and causal Sobel sequences, respectively, having later rejected this proposed separation. We then evaluated a proposed felicity factor of \textcite{Lewis2018}: that the distance in world similarity between the closest $\phi$-worlds and the closest $\phi\land\psi$-worlds directly correlates to reverse Sobel sequence felicity. To this end, we conducted an experiment that showed the following results: First, two populations exist. The more agreeable population rated reverse Sobel sequences with more dissimilar worlds as more acceptable than ones with less dissimilar worlds. The less agreeable population did not distinguish between the two types of reverse Sobel sequences. Both populations, however, rated reverse Sobel sequences as acceptable when the possibility of $\psi$ is epistemically excludable.  In the end, we postulated that world dissimilarity is not a deciding factor beyond the fact that the two sets of worlds must simply be of different degrees of world similarity---the more agreeable population was simply more willing to deem $\psi$ epistemically excludable for the sake of cooperativity. In the end, in this chapter, we isolated the following empirical factors: the presence of contrastive stress (either on an overtly different lexical item or on the auxiliary verb of the second antecedent), whether or not $\phi$ and $\psi$ are causally independent from one another, whether or not $\psi$ was counterfactual by nature, and whether or not the possibility of $\psi$ can be epistemically excluded via either overt or covert means.

In \refch{pragmatics-SS}, we then constructed a formal model for reverse Sobel sequences that centred around contrastive stress. We argued that contrastively stressed auxiliary verbs in the antecedent of conditionals should be treated as a type of contrastively stressed bound pro-forms, where the auxiliary verb is a bound pro-world that is bound by their conditional's domain of quantification. In order for the contrast to then be successful, the two domains in question must be entirely disjoint \parencite{Sauerland1998,Sauerland1999,Sauerland2000,Jacobson2000,Jacobson2004,Mayr2012}. We then showed that this requirement automatically explains why causal reverse Sobel sequences are typically infelicitous, assuming a world similarity metric along the lines of \textcite{Bennett2003} and \textcite{Arregui2009}. We also postulated that non-counterfactual reverse Sobel sequences are infelicitous for identical reasons, predicting that non-counterfactual conditionals quantify over a single degree of world similarity. We then showed that the epistemic exclusion of $\psi$ as a possibility rescues any reverse Sobel sequence by virtue of eliminating all $\psi$-worlds from otherwise non-empty domains of quantification before comparing whether or not the set of $\phi$-worlds and the set of $\phi\land\psi$-worlds are disjoint. We motivated the need for contrastive stress (and the resulting domain comparison) by either obligatory modal subordination for the variably-strict account or the expanding modal horizon for the (semi-)dynamic strict account. Finally, we concluded that our model is able to account for the presented empirical data on reverse Sobel sequences equally well with either approach to conditionals. We therefore tentatively eliminated reverse Sobel sequences from being a deciding factor in the debate between the two approaches to conditional semantics.

In \refch{ippolito}, we compared the model we constructed in \refch{pragmatics-SS} to another account for reverse Sobel sequence infelicity proposed by \textcite{Ippolito2020}. We showed that the two approaches are not completely incompatible with one another, how we can use \citepos{Ippolito2020} model to formalise the process of imprecision and precisification, and how \textcite{Ippolito2020} can use our model to independently motivate her claim why the two domains of quantification ought to be disjoint from one another---in addition to also then being able to explain the similar flavour of non-counterfactual and causal reverse Sobel sequences as well as why contrastive stress is needed in the first place for reverse Sobel sequence felicity.

In general, our thesis has advanced a number of different semantic topics. We advanced the operator-based approach by providing an explanation for negative bias in polar NPI questions, showed how the two approaches interact with the {\scshape even} that licenses NPIs, identified the factors that determine reverse Sobel sequence (in-)felicity (an issue that has long since confounded the role they play in formal semantics and pragmatics), and provided a formal account that accurately predicts all of the currently available empirical data on reverse Sobel sequences (not only their felicity distribution but also the pragmatic flavour that some of them exhibit).

For the general aim of our thesis, furthering the debate between the two competing approaches to conditional semantics, we would overall conclude that either approach can by and large accurately account for all of the data presented in this thesis. It is true, at this point in time, that the (semi-)dynamic strict semantic approach advocated for by \textcite{Fintel2001} and \textcite{Gillies2007} has a slight explanatory advantage over the variably-strict approach. We predict, however, that this advantage can likely be eliminated with further modifications of the variably-strict approach.

For future research, we propose that the empirical distribution of NPIs/\textit{even ONE} in conditionals should be empirically settled via a separate experiment.