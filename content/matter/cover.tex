
\title{The Semantics and Pragmatics of Counterfactual Sentences at the Discourse Level: Sobel-Sequences and the Licensing of Negative Polarity Items}

\author{David Krassnig}
\department{Department of Linguistics}

\degree{Doctor of Philosophy}

\degreemonth{May}
\degreeyear{2023}
\thesisdate{MAY 22, 2023}


\supervisor{Mar\'ia Isabel Romero Sang\"uesa}{Professor}

\chairman{Someone B. Chairman}{Chairman, Department Committee on Graduate Theses}

\definecolor{konstanz}{RGB}{0,169,224}
\setulcolor{konstanz}
\setul{3pt}{1.5pt}
\begin{titlepage}\doublespacing
\begin{adjustwidth}{-4mm}{-4mm}
\begin{center}
\textbf{\huge The Semantics and Pragmatics of Counterfactual Sentences at the Discourse Level: Sobel-Sequences and the Licensing of Negative Polarity Items}
\end{center}
\end{adjustwidth}

\textbf{\large Doctoral Thesis for Obtaining the Academic Degree\\ Doctor of Philosophy (Dr.~phil.)}

submitted by\\{\large \textit{David Krassnig}}%

at the\\\vspace{-2.5mm}
\includegraphics[scale=0.25]{content/graphics/unikonstanz_signet}


{\large\scshape Faculty of Humanities\\
Department of Linguistics}


Konstanz, 2023
\end{titlepage}
\thispagestyle{empty}
\hbox{}\vfill\begin{flushleft}
\noindent\textbf{{Doctoral Thesis of the University of Konstanz}}

\enlargethispage{3\baselineskip}\noindent\textbf{Day of the oral examination:} 22\textsuperscript{nd} May 2023\\
\textbf{Referee:} Prof. Dr. Maribel Romero\\
\textbf{Referee:} Prof. Dr. Regine Eckardt\\
\textbf{Referee:} Asst. Prof. Dr. María Biezma
\end{flushleft}


\cleardoublepage%
\thispagestyle{empty}
\phantomsection\addcontentsline{toc}{chapter}{\protect Front Matter}%
\hbox{}\vskip 10mm
\begin{flushright}\doublespacing
\textit{\LARGE Life Before Death}\\
\textit{\LARGE Strength Before Weakness}\\
\textit{\LARGE Journey Before Destination}\\\vskip -2mm
\resizebox{81.4mm}{!}{\textbf{from Brandon Sanderson's \textit{The Stormlight Archive}}}
\end{flushright}
\enlargethispage{3\baselineskip}\vfill
\begin{center}\singlespacing
\includesvg[width=33.33mm]{content/graphics/newestvector_hatched.svg}
\end{center}

\cleardoublepage
\section*{Acknowledgements}\addcontentsline{toc}{section}{\protect Acknowledgements}%
First and foremost, I would like to express my deepest gratitude to the best supervisor of all possible worlds, Maribel Romero. If it were not for her, I very likely would not have become a semanticist: Having attended my first seminar on semantics with her as an instructor, I was not only captivated by the rigorous logic of formal semantics but also by her practically endless enthusiasm for her field of study. I could not have wished for a better supervisor, both in a professional and in an interpersonal capacity.

I would also like to express my deepest appreciation to Mar\'ia Biezma, my first second supervisor. She always made sure that I do not consider any given issue in isolation but in the broader context of language, always keeping the bigger picture in mind. She is also largely at fault for me warming up to formal pragmatics, when I originally studied only formal semantics.

I am also very thankful to my second second supervisor, Regine Eckardt. She finished the job that Mar\'ia had started with regards to formal pragmatics. Special thanks to her are due for agreeing to supervise my thesis committee.

I would further like to extend my sincere thanks to my predecessor, Andreas Walker. Having been the only PhD candidate in the field of semantics at the beginning of his studies at the University of Konstanz , he greatly influenced the culture of the emergent circle of fledgling semanticists in Konstanz. The friendly, egalitarian, and ridiculously supportive environment that I experienced in my time as a PhD student there is, in part, a testament to his tremendously caring nature and his dedication to leaving this world a better place.

Along this line of thought, I would also like to express my gratitude to my fellow PhD candidate and long-time office mate, Mark-Matthias Zymla. We had countless conversations on the nature of language and meaning, and even more discussions about the most inane and ridiculous things. He is, I think, partly responsible for me having kept my sanity during the often stressful times leading up to this dissertation.

Naturally, I also give many thanks to all of my other fellow semanticists and pragmaticists at the University of Konstanz. This includes but is not limited to Moritz Meßmer, Felicitas Enders, Erlinde Meertens, Gisela Disselkamp (geb. Grohne), Felix Frühauf, Maryam Mohammadi, Ramona Baumgartner (geb. Wallner), Kajsa Dj\"{a}rv, Arno Goebel, Vasiliki Erotokritou, Yvonne Viesel, Katerina Kalouli, Sven Lauer, Brian Leahy, Eva Csipak, Ryan Bochnak, Todor Koev, and Chen-An Chang.

There are also numerous linguists outside of this university to whom I would like to extend my deepest thanks for meeting with me and/or for providing me with their very insightful comments on my work, including but not limited to Prof.~Dr.~Luka Crni\v{c}, Prof.~Dr.~Kai von Fintel, Prof.~Dr.~Peter Klecha, Prof.~Dr.~Lucas Champollion, and Prof.~Dr.~Irene Heim.

I would also like to thank the research group \enquote{What if? On the Epistemological, Pragmatic, Psychological and Cultural Significance of Counterfactual Thinking} for funding large parts of my research. I would also like to thank each of the research group members for their support, comments, and help over the years. Here, very special thanks go to Prof. Dr. Wolfgang Spohn who masterfully organised and advocated for this research group.

But I must reserve my greatest of thanks to my family: to Stefan, the best big brother who always has my back; to my mother and to my father, Roswitha and Karl-Heinz, who always loved and supported me without condition or reservation; to Blacky, for having been my best friend for nearly two decades, may he rest in peace; to Anna, for being as close to a twin as a cousin can be; to my grandmothers, Oma K\"athe and Oma Resi, for never once making me doubt all the best stereotypes about grandmothers; to my mother-in-law, Mary, for taking me in as if I was her own and, in turn, discrediting all stereotypes about mother-in-laws; to my sisters-in-law, Leyla and Melissa, for being the crazy little sisters I never had; and to my father-in-law, for all he has sacrificed for his family.

Finally, to my wife, Sarah Ceylan Krassnig: You are the great love of my life and \enquote{If I had not met you, I would never have been complete} is the only counterfactual conditional that I would never fathom to be part of a felicitous Sobel sequence.
\cleardoublepage
\section*{Abstract (English)}\addcontentsline{toc}{section}{\protect Abstract (English)}%
In this thesis, I explore the question whether conditionals can be more accurately modelled with a \mbox{(semi-)}dynamic strict \parencite{Fintel2001,Gillies2007} or a variably-strict \parencite{Stalnaker1968,Lewis1973} semantics. To this end, I examine two linguistic phenomena that are associated with conditionals: (weak) negative polarity items and (reverse) Sobel sequences. Negative polarity items are words with a varyingly restrictive distribution that mostly (but not exclusively) occur in negative contexts (e.g., \textit{any} and \textit{ever}). Sobel sequences are conditional sequences that adhere to the pattern of \enquote{If $\phi$, $\chi$; but if $\phi\land\psi$, $\neg\chi$}. 

For negative polarity items, I rule out the traditional account that stipulated that they are licensed by downward monotone environment \parencite{Fauconnier1975a,Fauconnier1975b,Ladusaw1980} due to the contextually changing licensing of negative polarity items in non-monotone environments \parencite{Crnic2011}. This account would have restricted us to a (semi-)dynamic conditional semantics, which is (Strawson) downward monotone by nature, as the variably-strict semantics is non-monotone by nature. I then evaluate an alternative approach to negative polarity item licensing: the operator-based approach that stipulates that negative polarity items are licensed by a covert \textit{even}-like operator that imposes a probability-based scalar presupposition on its associated sentences \parencite[see, amongst others,][]{Crnic2014-dogma,Crnic2014-nm}. I improve the viability of this model by solving an issue that this model had with deriving a difference in question bias between questions containing unfocused weak negative polarity items and questions that contain the expression \textit{even \MakeUppercase{one}} \parencite[see, amongst others,][]{Crnic2014-dogma,Crnic2014-nm,Krassnig2018,Jeong2021,Jeong2022}. I then evaluate the interaction of this licensing model with a variably-strict conditional semantics and with a (semi-)dynamic strict conditional semantics. I show that the latter has a slight explanatory advantage over the former, being able to account for all of the known data, but conclude that this advantage might be mitigated by future research.

(Reverse) Sobel sequences have traditionally been considered another key piece in the debate on how to accurately model conditionals because regularly ordered Sobel sequences are felicitous but reverse Sobel sequences are infelicitous. This status originally spoke in favour of a (semi-)dynamic strict semantics, which was designed to predict reverse Sobel sequences to be infelicitous, because the variably-strict semantics predicts that either sequence type should be felicitous. However, recent developments have shown that some reverse Sobel sequences might be consistently rendered felicitous \parencite{Moss2012}, prompting a return to the variably-strict approach to conditionals with the addition of different selective pragmatic mechanisms that rule out some but not all reverse Sobel sequences \parencite{Moss2012,Klecha2014,Klecha2015,Lewis2018,Krassnig2017,Krassnig2020,Krassnig2022,Ippolito2020}. I survey the current empirical landscape on reverse Sobel sequences and improve upon it by conducting a reverse Sobel sequence felicity experiment. I conclude that the crucial factor for reverse Sobel sequence felicity is contrastive stress in the antecedent of the $\phi$-conditional and isolate a number of highly influential sub-factors for felicity that were partly already proposed in the literature in isolation: namely, (i) counterfactuality, (ii) a lack of a causal link between $\phi$ and $\psi$, and (iii) the use of overt or covert exclusion of $\psi$ as an epistemic possibility (the last of which acts as a felicity rescue operation for contrastively stressed reverse Sobel sequences). I propose a pragmatic model constructed around the effect of contrastive stress and show how this model is able to account for all of the known empirical data on reverse Sobel sequences via the interaction of contrastive stress and the aforementioned factors (where all components can be independently motivated by other phenomena in the literature). Crucially, I show that this proposed supererogatory pragmatic mechanism may be combined with either the variably-strict model or the (semi-)dynamic strict model without a change in the predicted felicity distribution (though the former requires a few additional independently motivated pragmatic mechanisms to do so).

In the end, I conclude that the (semi-)dynamic strict account appears to have a slight explanatory advantage over the variably-strict approach overall, but I also conclude that this advantage can likely be eliminated with further modifications to the variably-strict approach, rendering both approaches equally viable with respect to (reverse) Sobel sequences and negative polarity items.


\cleardoublepage
\section*{Abstract (German)}\addcontentsline{toc}{section}{\protect Abstract (German)}%
\begin{otherlanguage}{ngerman}
In dieser Dissertation untersuche ich die Frage, ob Konditionale eher mit einer \mbox{(semi-)}\linebreak dynamisch strikten \parencite{Fintel2001,Gillies2007} oder einer variabel-strikten Semantik \parencite{Stalnaker1968,Lewis1973} modelliert werden können. Zu diesem Zwecke untersuche ich zwei linguistische Phänomene, die mit Konditionalen assoziiert werden: (schwache) negative Polaritätselemente und (umgekehrte) Sobel-Sequenzen. Negative Polaritätselemente sind Worte mit einer unterschiedlich stark restriktiven Verteilung, die hauptsächlich (aber nicht exklusiv) in negativen Kontexten auftauchen (z.B. \textit{any} und \textit{ever} im Englischen). Sobel-Sequenzen sind Konditionalsequenzen, die dem Muster \glqq Falls $\phi$, $\chi$; aber falls $\phi\land\psi$, $\neg\chi$\grqq\ entsprechen.
    
Für negative Polaritätselemente schließe ich, aufgrund der kontextuell-variablen Lizensierung von negativen Polaritätselementen in nicht-monotonen Umgebungen \parencite{Crnic2011},  die traditionelle Lizensierungstheorie aus, welche davon ausgeht, dass negative Polaritätselemente durch abwärts monotone Umgebung lizensiert werden \parencite{Fauconnier1975a,Fauconnier1975b,Ladusaw1980}. Da die variabel-strikte semantik nicht-\linebreak monotoner Natur ist, hätte uns diese Theorie auf eine (semi-)dynamische strikte Semantik, welche (Strawson) abwärts monotoner Natur ist, eingeschränkt. Ich evaluiere dann einen alternativen Ansatz zur Lizensierung von negativen Polaritätselementen: der Operator-basierende Ansatz, welcher davon ausgeht, dass negative Polaritätselemente durch einen verdeckten \textit{even}-ähnlichen Operator lizensiert werden, der den verbundenen Sätzen, in denen er vorkommt, eine auf Wahrscheinlichkeiten basierende Skalarpräsupposition auferlegt \parencite[siehe, unter anderem][]{Crnic2014-dogma,Crnic2014-nm}. Ich verbessere die Viabilität dieses Modells dadurch, dass ich ein Problem löse, das dieses Modell hatte: die Herleitung des Unterschiedes in negativer Voreingenommenheit zwischen Fragen, die ein unfokusiertes schwaches negatives Polaritätselement beinhalten, und Fragen, die den Ausdruck \textit{even ONE}\footnote{Das nächstgelegene deutsche Äquivalent zu \textit{even ONE} wäre \textit{auch nur EIN}.} beinhalten \parencite[vgl.][]{Crnic2014-dogma,Crnic2014-nm,Krassnig2018,Jeong2021,Jeong2022}. Dann evaluiere ich die Interaktion dieses Lizensierungsmodells mit einer variabel-strikten Semantik und einer (semi-)dynamisch strikten Semantik. Ich zeige, dass die Letztere einen schwachen explanativen Vorteil gegenüber der Ersteren hat, dass die Letztere alle bekannten Fakten erklären kann, aber auch dass dieser Vorteil eventuell durch zukünftige Forschung abgeschwächt werden könnte.
    
(Umgekehrte) Sobel-Sequenzen werden traditionell als weiterer Schlüsselfaktor in der Debatte darüber, wie Konditionale am besten modelliert werden können, betrachtet, da regulär geordnete Sobel-Sequenzen als gelungen und umgekehrte Sobel-Sequenzen als mißlungen gelten. Da die variabel-strikte Semantik vorhersagt, dass beide Sequenztypen gelungen sein sollten, sprach dieser Status ursprünglich für eine \mbox{(semi-)}dynamische strikte Semantik, die dafür entworfen wurde, umgekehrte Sobel-\linebreak Sequenzen als mißlungen vorherzusagen. Neuere Entwicklungen haben jedoch aufgezeigt, dass einige umgekehrte Sobel-Sequenzen konsistent als gelungen eingestuft werden können \parencite[vgl.][]{Moss2012}. Dies führte zu einer Rückkehr zum variabel-strikten Ansatz zu Konditionalen mit der Hinzufügung verschiedener selektiver pragmatischer Mechanismen, die einige aber nicht alle umgekehrte Sobel-Sequenzen ausschließen \parencite[vgl.][]{Moss2012,Klecha2014,Klecha2015,Lewis2018,Krassnig2017,Krassnig2020,Krassnig2022,Ippolito2020}. Ich untersuche die Genauigkeit der aktuellen empirischen Landschaft von umgekehrten Sobel-Sequenzen und verbessere diese durch die Durchführung eines  Experiments, welche umgekehrte Sobel-Sequenzen auf ihre Gelungenheit untersucht. Dann schlussfolgere ich, dass der entscheidende Faktor für die Gelungenheit von umgekehrten Sobel-Sequenzen der kontrastive Stress im Antezedenten der $\phi$-Konditionale ist und isoliere eine Anzahl von einflussreichen Unterfaktoren für Gelungenheit, die teilweise bereits in der Literatur isoliert vorgeschlagen wurden: nämlich (i)~Kontrafaktualität, (ii)~das Fehlen eines kausalen Zusammenhangs zwischen $\phi$ und $\psi$ und (iii)~die Verwendung eines offenen oder verdeckten Ausschlusses von $\psi$ als epistemische Möglichkeit (Letzteres fungiert als Gelungenheit-Rettungsoperation für kontrastiv betonte umgekehrte Sobel-Sequenzen). Ich schlage ein pragmatisches Modell vor, das um den Effekt des kontrastiven Stresses herum aufgebaut ist, und zeige auf, wie dieses Modell durch die Interaktion zwischen dem kontrastiven Stress und den vorher genannten Faktoren alle bekannten empirischen Daten zu umgekehrten Sobel-Sequenzen korrekt vorhersagen kann (wobei alle Komponenten durch andere Phänomene in der Literatur unabhängig motiviert werden können). Schlussendlich zeige ich, dass dieser vorgeschlagene pragmatische Mechanismus entweder mit dem variabel-strikten Modell oder mit dem (semi-)dynamischen strikten Modell kombiniert werden kann, ohne dass sich die vorhergesagte Gelungenheitsdistribution verändert (wobei letztere Semantik einige zusätzliche unabhängig motivierte pragmatische Mechanismen erfordert).
    
Am Ende schlussfolgere ich, dass das (semi-)dynamische strikte Modell insgesamt einen leichten explanativen Vorteil gegenüber dem variabel-strikten Ansatz hat. Ich schlussfolgere aber auch, dass dieser Vorteil wahrscheinlich durch weitere Modifikationen des variabel-strikten Ansatzes eliminiert werden kann, wodurch beide Ansätze in Bezug auf (umgekehrte) Sobel-Sequenzen und negative Polartitäts-Elemente gleichermaßen tragfähig werden würden.
\end{otherlanguage}

\cleardoublepage
\section*{Abstract (Esperanto)}\addcontentsline{toc}{section}{\protect Abstract (Esperanto)}%
\begin{otherlanguage}{esperanto}\setquotestyle{esperanto}
En ĉi tiu disertaĵo, mi esploras la demandon ĉu oni povas modeli kondiĉfrazojn pli precize per (duon-)dinamika strikta semantiko \parencite{Fintel2001,Gillies2007} aŭ variabla-strikta semantiko \parencite{Stalnaker1968,Lewis1973}. Por tio, mi ekzamenas du lingvajn fenomenojn kiuj estas asociitaj kun kondiĉfrazoj: (malfortajn) negativajn polusaĵojn kaj (inversajn) Sobel-sekvencojn. Negativaj polusaĵoj estas vortoj kun varie restriktiva distribuo kiuj plejparte (sed ne ekskluzive) aperas en negativaj kuntekstoj (ekz. \textit{any} kaj \textit{ever} en la angla). Sobel-sekvencoj estas kondiĉfrazaj sekvencoj kiuj obeas al la ŝablono de \enquote{Se $\phi$, $\chi$; sed se $\phi\land\psi$, $\neg\chi$}.

Por negativaj polusaĵoj, mi forĵetas la tradician klarigon, kiu asertis ke negativaj polusaĵoj permesiĝas per malsuprene monotonaj medioj \parencite{Fauconnier1975a,Fauconnier1975b,Ladusaw1980}, pro la kuntekste ŝanĝiĝanta permesigeco de negativaj polusaĵoj en ne-monotonaj medioj \parencite{Crnic2011}. Tiu klarigo restriktus nin al (duon-)dinamika kondiĉfraza semantikoj, kiu estas (Strawson-e) malsupre monotona medie, ĉar la variabla-strikta semantiko estas ne-monotona medie. Mi tiam taksas alternativan aliron al la permesigeco de negativaj polusaĵoj: la operatora aliro kiu asertas ke negativaj polusaĵoj permesiĝas per kaŝa \textit{eĉ}-simila operatoro kiu aldonas probablan skalisan antaŭsupozon al siaj asociitaj frazoj \parencite[vidu, inter aliajn,][]{Crnic2014-dogma,Crnic2014-nm}. Mi plibonigas la vivipoveco de ĉi tiu modelo per solvado problemon kiun ĉi tiu modelo havis pri derivado la malsamon de demando-inkliniĝo inter demandoj kiuj enhavas ne-fokusigatan malfortan negativan polusaĵon kaj demandoj kiuj enhavas la esprimon \textit{eĉ unu} \parencite[vidu, inter aliajn,][]{Crnic2014-dogma,Crnic2014-nm,Krassnig2018,Jeong2021,Jeong2022}. Poste, mi evalvas la interagon de ĉi tiu permesiga modelo kun variabla-strikta kondiĉfraza semantiko kaj kun (duon-)dinamika strikta kondiĉfraza semantiko. Mi montras ke la lasta havas iomajn klarigajn avantaĝojn kontraŭ la unua, ke ĝi povas klarigi ĉiujn konatajn datumojn, sed konkludas ke ĉi tiu avantaĝo povus esti malgrandigita per estonta esplorado.
    
(Inversaj) Sobel-sekvencoj estas tradicie konsideritaj aliaj ĉefaj punktoj en la diskuto pri la preciza modeligo de kondiĉfrazoj, ĉar regule ordigitaj Sobel-sekvencoj estas feliĉumaj, sed inversaj Sobel-sekvencoj estas malfeliĉumaj. Tiu statuso unue favoris (duon-)dinamikan striktan semantikon, kiu estis kreita por antaŭdiri ke inversaj Sobel-sekvencoj estu malfeliĉumaj, ĉar la variabla-strikta semantiko antaŭdiras ke ambaŭ sekvenctipoj estu feliĉumaj. Tamen, lastatempaj evoluoj montris, ke kelkaj inversaj Sobel-sekvencoj povas esti konstante feliĉumaj \parencite{Moss2012}, instigante revenon al la variabla-strikta aliro al kondiĉfrazoj kun aldonaj elektivaj pragmatikaj mekanismoj kiuj elĵetas iujn sed ne ĉiujn inversajn Sobel-sekvencojn \parencite{Moss2012,Klecha2014,Klecha2015,Lewis2018,Krassnig2017,Krassnig2020,Krassnig2022,Ippolito2020}. Mi esploras la nunan empirian pejzaĝon pri inversaj Sobel-sekvencoj kaj plibonigas ĝin per realigo de esperimento pri feliĉumeco de inversaj Sobel-sekvencoj. Mi konkludas, ke la decida faktoro por la aprobo de inversaj Sobel-sekvencoj estas kontrasta streso en la antaŭaĵo de la $\phi$-kondiĉfrazo kaj apartigas kelkajn tre influajn subfaktorojn por aprobo, kiuj estis parte jam proponitaj en la literaturo izole: nome, (i) kontrafaktualeco, (ii) manko de kaŭza ligilo inter $\phi$ kaj $\psi$, kaj (iii) la uzo de malkaŝa aŭ kaŝa ekskludo de $\psi$ kiel epistema ebleco (la lasta el kiuj agas kiel feliĉumeco-savilo por kontraste streseblaj inversaj Sobel-sekvencoj). Mi proponas pragmatikan modelon kiu estas konstruita ĉirkaŭ la efiko de kontrasta streso kaj mi montras kiel ĉi tiu modelo kapablas klarigi ĉiujn konatajn empiriajn datumojn pri inversaj Sobel-sekvencoj per la interago de kontrasta streso kaj la antaŭdiraj faktoroj (kie ĉiuj komponantoj povas esti sendepende motivitaj per aliaj fenomenoj en la literaturo). Grave, mi montras ke ĉi tiu proponita aldona pragmatika mekanismo povas esti kombinita aŭ kun la variabla-strikta modelo aŭ kun la (duon-)dinamika strikta modelo sen ŝanĝo en la antaŭdirita feliĉumeco-distribuo (kvankam la unua postulas kelkajn aldonajn sendepende motivitajn pragmatikajn mekanismojn por plenumi tion).
    
Fine, mi konkludas ke la (duon-)dinamika strikta klarigo ŝajnas havi iometan klarigan avantaĝon kontraŭ la variabla-strikta aliro en la tutaĵo, sed mi ankaŭ konkludas ke ĉi tiu avantaĝo verŝajne povas esti eliminata per pluaj modifoj al la variabla-strikta aliro, farante ambaŭ aliroj egale taŭgaj rilate al (inversaj) Sobel-sekvencoj kaj negativaj polusaĵoj.
\end{otherlanguage}

\cleardoublepage
